%% Generated by Sphinx.
\def\sphinxdocclass{report}
\documentclass[letterpaper,10pt,english]{sphinxmanual}
\ifdefined\pdfpxdimen
   \let\sphinxpxdimen\pdfpxdimen\else\newdimen\sphinxpxdimen
\fi \sphinxpxdimen=.75bp\relax
\ifdefined\pdfimageresolution
    \pdfimageresolution= \numexpr \dimexpr1in\relax/\sphinxpxdimen\relax
\fi
%% let collapsible pdf bookmarks panel have high depth per default
\PassOptionsToPackage{bookmarksdepth=5}{hyperref}

\PassOptionsToPackage{warn}{textcomp}
\usepackage[utf8]{inputenc}
\ifdefined\DeclareUnicodeCharacter
% support both utf8 and utf8x syntaxes
  \ifdefined\DeclareUnicodeCharacterAsOptional
    \def\sphinxDUC#1{\DeclareUnicodeCharacter{"#1}}
  \else
    \let\sphinxDUC\DeclareUnicodeCharacter
  \fi
  \sphinxDUC{00A0}{\nobreakspace}
  \sphinxDUC{2500}{\sphinxunichar{2500}}
  \sphinxDUC{2502}{\sphinxunichar{2502}}
  \sphinxDUC{2514}{\sphinxunichar{2514}}
  \sphinxDUC{251C}{\sphinxunichar{251C}}
  \sphinxDUC{2572}{\textbackslash}
\fi
\usepackage{cmap}
\usepackage[T1]{fontenc}
\usepackage{amsmath,amssymb,amstext}
\usepackage{babel}



\usepackage{tgtermes}
\usepackage{tgheros}
\renewcommand{\ttdefault}{txtt}



\usepackage[Bjarne]{fncychap}
\usepackage{sphinx}

\fvset{fontsize=auto}
\usepackage{geometry}


% Include hyperref last.
\usepackage{hyperref}
% Fix anchor placement for figures with captions.
\usepackage{hypcap}% it must be loaded after hyperref.
% Set up styles of URL: it should be placed after hyperref.
\urlstyle{same}

\addto\captionsenglish{\renewcommand{\contentsname}{Contents:}}

\usepackage{sphinxmessages}
\setcounter{tocdepth}{0}



\title{hpsOpenMM Documentation}
\date{November 07, 2022}
\release{v1.3}
\author{hpsOpenMM}
\newcommand{\sphinxlogo}{\vbox{}}
\renewcommand{\releasename}{Release}
\makeindex
\begin{document}

\ifdefined\shorthandoff
  \ifnum\catcode`\=\string=\active\shorthandoff{=}\fi
  \ifnum\catcode`\"=\active\shorthandoff{"}\fi
\fi

\pagestyle{empty}
\sphinxmaketitle
\pagestyle{plain}
\sphinxtableofcontents
\pagestyle{normal}
\phantomsection\label{\detokenize{index::doc}}


\sphinxAtStartPar
This documentation is currently being generated as we finalize

\sphinxstepscope


\chapter{Introduction}
\label{\detokenize{modules/introduction:introduction}}\label{\detokenize{modules/introduction::doc}}
\sphinxAtStartPar
The hpsOpenMM model is a Python library that offers flexibility to set up coarse\sphinxhyphen{}grained simulation of IDP using the MD framework of OpenMM toolkit.
The codebase is based on sbmOpenMM scripts.
It automates the creation of \sphinxcode{\sphinxupquote{openmm.system}} classes that contain the necessary force field parameters to run molecular dynamics simulations using a protein structure as the only necessary inputs.

\sphinxAtStartPar
hpsOpenMM is divided in four main classes:
\begin{enumerate}
\sphinxsetlistlabels{\arabic}{enumi}{enumii}{}{.}%
\item {} 
\sphinxAtStartPar
\sphinxcode{\sphinxupquote{system}}

\item {} 
\sphinxAtStartPar
\sphinxcode{\sphinxupquote{models}}

\item {} 
\sphinxAtStartPar
\sphinxcode{\sphinxupquote{dynamics}}

\item {} 
\sphinxAtStartPar
\sphinxcode{\sphinxupquote{geometry}}

\end{enumerate}


\bigskip\hrule\bigskip


\sphinxAtStartPar
\sphinxcode{\sphinxupquote{system}}, is the main class that holds all the methods to define,
modify and create CG system to be simulated with OpenMM.
Class inheritance from \sphinxcode{\sphinxupquote{openmm.system}} with some more attributes for hpsOpenMM.


\bigskip\hrule\bigskip


\sphinxAtStartPar
\sphinxcode{\sphinxupquote{models}}, class contains set of models, each model contains a collection of sequence of commands
to build model, allows to easily set up CG models.


\bigskip\hrule\bigskip


\sphinxAtStartPar
\sphinxcode{\sphinxupquote{dynamics}} class auto read the parameter controls, build the model and run simulation.


\bigskip\hrule\bigskip


\sphinxAtStartPar
\sphinxcode{\sphinxupquote{geometry}}, contains methods to calculate the geometrical parameters from the input structures.
It’s not useful in current need of simulation method.


\bigskip\hrule\bigskip


\sphinxAtStartPar
The library is open\sphinxhyphen{}source and offers flexibility to simulate IDPs.

\sphinxstepscope


\chapter{Simulation control options}
\label{\detokenize{usage/simulation_control:simulation-control-options}}\label{\detokenize{usage/simulation_control::doc}}
\sphinxAtStartPar
An example of how config file of simulation looks like.

\begin{sphinxVerbatim}[commandchars=\\\{\}]
\PYG{p}{[}\PYG{n}{OPTIONS}\PYG{p}{]}
\PYG{n}{md\PYGZus{}steps} \PYG{o}{=} \PYG{l+m+mi}{30\PYGZus{}000} \PYG{c+c1}{\PYGZsh{} number of steps}
\PYG{n}{dt} \PYG{o}{=} \PYG{l+m+mf}{0.01} \PYG{p}{;} \PYG{n}{time} \PYG{n}{step} \PYG{o+ow}{in} \PYG{n}{ps}
\PYG{n}{nstxout} \PYG{o}{=} \PYG{l+m+mi}{1000} \PYG{p}{;} \PYG{n}{number} \PYG{n}{of} \PYG{n}{steps} \PYG{n}{to} \PYG{n}{write} \PYG{n}{checkpoint} \PYG{o}{=} \PYG{n}{nstxout}
\PYG{n}{nstlog} \PYG{o}{=} \PYG{l+m+mi}{1000} \PYG{p}{;} \PYG{n}{number} \PYG{n}{of} \PYG{n}{steps} \PYG{n}{to} \PYG{n+nb}{print} \PYG{n}{log}
\PYG{n}{nstcomm} \PYG{o}{=} \PYG{l+m+mi}{100} \PYG{p}{;} \PYG{n}{frequency} \PYG{k}{for} \PYG{n}{center} \PYG{n}{of} \PYG{n}{mass} \PYG{n}{motion} \PYG{n}{removal}
\PYG{p}{;} \PYG{n}{select} \PYG{n}{HPS} \PYG{n}{model}\PYG{p}{,} \PYG{n}{available} \PYG{n}{options}\PYG{p}{:} \PYG{n}{hps\PYGZus{}kr}\PYG{p}{,} \PYG{n}{hps\PYGZus{}urry}\PYG{p}{,} \PYG{o+ow}{or} \PYG{n}{hps\PYGZus{}ss}
\PYG{n}{model} \PYG{o}{=} \PYG{n}{hps\PYGZus{}urry}

\PYG{p}{;} \PYG{n}{control} \PYG{n}{temperature} \PYG{n}{coupling}
\PYG{n}{tcoupl} \PYG{o}{=} \PYG{n}{yes}
\PYG{n}{ref\PYGZus{}t} \PYG{o}{=} \PYG{l+m+mi}{310} \PYG{p}{;} \PYG{n}{Kelvin}\PYG{o}{\PYGZhy{}} \PYG{n}{reference} \PYG{n}{temperature}
\PYG{n}{tau\PYGZus{}t} \PYG{o}{=} \PYG{l+m+mf}{0.01} \PYG{p}{;} \PYG{n}{ps}\PYG{o}{\PYGZca{}}\PYG{o}{\PYGZhy{}}\PYG{l+m+mi}{1}

\PYG{p}{;}\PYG{n}{pressure} \PYG{n}{coupling}
\PYG{n}{pcoupl} \PYG{o}{=} \PYG{n}{yes}
\PYG{n}{ref\PYGZus{}p} \PYG{o}{=} \PYG{l+m+mi}{1}
\PYG{n}{frequency\PYGZus{}p} \PYG{o}{=} \PYG{l+m+mi}{25}

\PYG{p}{;} \PYG{n}{Periodic} \PYG{n}{boundary} \PYG{n}{condition}\PYG{p}{:} \PYG{k}{if} \PYG{n}{pcoupl} \PYG{o+ow}{is} \PYG{n}{yes} \PYG{n}{then} \PYG{n}{pbc} \PYG{n}{must} \PYG{n}{be} \PYG{n}{yes}\PYG{o}{.}
\PYG{n}{pbc} \PYG{o}{=} \PYG{n}{yes}
\PYG{p}{;} \PYG{k}{if} \PYG{n}{pbc}\PYG{o}{=}\PYG{n}{yes}\PYG{p}{,} \PYG{n}{then} \PYG{n}{use} \PYG{n}{box\PYGZus{}dimension} \PYG{n}{option} \PYG{n}{to} \PYG{n}{specify} \PYG{n}{box\PYGZus{}dimension} \PYG{o}{=} \PYG{n}{x} \PYG{o+ow}{or} \PYG{p}{[}\PYG{n}{x}\PYG{p}{,} \PYG{n}{y}\PYG{p}{,} \PYG{n}{z}\PYG{p}{]}\PYG{p}{,} \PYG{n}{unit} \PYG{n}{of} \PYG{n}{nanometer}
\PYG{n}{box\PYGZus{}dimension} \PYG{o}{=} \PYG{l+m+mi}{30} \PYG{p}{;} \PYG{p}{[}\PYG{l+m+mi}{30}\PYG{p}{,} \PYG{l+m+mi}{30}\PYG{p}{,} \PYG{l+m+mi}{60}\PYG{p}{]}

\PYG{p}{;} \PYG{n+nb}{input}
\PYG{n}{protein\PYGZus{}code} \PYG{o}{=} \PYG{n}{FUS\PYGZus{}100chains}
\PYG{n}{pdb\PYGZus{}file} \PYG{o}{=} \PYG{n}{FUS\PYGZus{}100chains}\PYG{o}{.}\PYG{n}{pdb}
\PYG{p}{;} \PYG{n}{output}
\PYG{n}{checkpoint} \PYG{o}{=} \PYG{n}{FUS\PYGZus{}100chains}\PYG{o}{.}\PYG{n}{chk}
\PYG{p}{;}\PYG{n}{Use} \PYG{n}{GPU}\PYG{o}{/}\PYG{n}{CPU}
\PYG{n}{device} \PYG{o}{=} \PYG{n}{GPU}
\PYG{p}{;} \PYG{n}{If} \PYG{n}{CPU} \PYG{o+ow}{is} \PYG{n}{specified}\PYG{p}{,} \PYG{n}{then} \PYG{n}{use} \PYG{n}{ppn} \PYG{n}{variable}
\PYG{n}{ppn} \PYG{o}{=} \PYG{l+m+mi}{4}
\PYG{p}{;}\PYG{n}{Restart} \PYG{n}{simulation}
\PYG{n}{restart} \PYG{o}{=} \PYG{n}{no}
\PYG{n}{minimize} \PYG{o}{=} \PYG{n}{yes} \PYG{p}{;}\PYG{k}{if} \PYG{o+ow}{not} \PYG{n}{restart}\PYG{p}{,} \PYG{n}{then} \PYG{n}{minimize} \PYG{n}{will} \PYG{n}{be} \PYG{n}{loaded}\PYG{p}{,} \PYG{n}{otherwise}\PYG{p}{,} \PYG{n}{minimize}\PYG{o}{=}\PYG{k+kc}{False}
\end{sphinxVerbatim}


\bigskip\hrule\bigskip



\section{General information}
\label{\detokenize{usage/simulation_control:general-information}}
\sphinxAtStartPar
Simulation parameters are input from \sphinxtitleref{.ini} file which is loaded by \sphinxtitleref{ConfigParser} module in Python.
The section title \sphinxcode{\sphinxupquote{{[}OPTIONS{]}}} is required, do not change section’s name.
\begin{itemize}
\item {} 
\sphinxAtStartPar
Comment can be inline or in new line, start with \sphinxtitleref{;} or \sphinxtitleref{\#}

\item {} 
\sphinxAtStartPar
Keyword and value can be separated by \sphinxtitleref{=} or \sphinxtitleref{:}

\end{itemize}


\section{Run control}
\label{\detokenize{usage/simulation_control:run-control}}
\begin{sphinxVerbatim}[commandchars=\\\{\}]
\PYG{n}{md\PYGZus{}steps}\PYG{p}{:}   \PYG{p}{(}\PYG{n}{long} \PYG{n+nb}{int}\PYG{p}{)}
            \PYG{p}{(}\PYG{l+m+mi}{1}\PYG{p}{)} \PYG{n}{Maximum} \PYG{n}{number} \PYG{n}{of} \PYG{n}{steps} \PYG{n}{to} \PYG{n}{integrate} \PYG{o+ow}{or} \PYG{n}{minimize}
\PYG{o}{\PYGZhy{}}\PYG{o}{\PYGZhy{}}\PYG{o}{\PYGZhy{}}\PYG{o}{\PYGZhy{}}\PYG{o}{\PYGZhy{}}\PYG{o}{\PYGZhy{}}\PYG{o}{\PYGZhy{}}\PYG{o}{\PYGZhy{}}\PYG{o}{\PYGZhy{}}\PYG{o}{\PYGZhy{}}\PYG{o}{\PYGZhy{}}\PYG{o}{\PYGZhy{}}\PYG{o}{\PYGZhy{}}\PYG{o}{\PYGZhy{}}\PYG{o}{\PYGZhy{}}\PYG{o}{\PYGZhy{}}\PYG{o}{\PYGZhy{}}\PYG{o}{\PYGZhy{}}\PYG{o}{\PYGZhy{}}\PYG{o}{\PYGZhy{}}\PYG{o}{\PYGZhy{}}\PYG{o}{\PYGZhy{}}\PYG{o}{\PYGZhy{}}\PYG{o}{\PYGZhy{}}\PYG{o}{\PYGZhy{}}\PYG{o}{\PYGZhy{}}\PYG{o}{\PYGZhy{}}\PYG{o}{\PYGZhy{}}\PYG{o}{\PYGZhy{}}\PYG{o}{\PYGZhy{}}\PYG{o}{\PYGZhy{}}\PYG{o}{\PYGZhy{}}\PYG{o}{\PYGZhy{}}\PYG{o}{\PYGZhy{}}\PYG{o}{\PYGZhy{}}\PYG{o}{\PYGZhy{}}\PYG{o}{\PYGZhy{}}\PYG{o}{\PYGZhy{}}\PYG{o}{\PYGZhy{}}\PYG{o}{\PYGZhy{}}\PYG{o}{\PYGZhy{}}\PYG{o}{\PYGZhy{}}\PYG{o}{\PYGZhy{}}\PYG{o}{\PYGZhy{}}\PYG{o}{\PYGZhy{}}\PYG{o}{\PYGZhy{}}\PYG{o}{\PYGZhy{}}\PYG{o}{\PYGZhy{}}\PYG{o}{\PYGZhy{}}\PYG{o}{\PYGZhy{}}\PYG{o}{\PYGZhy{}}\PYG{o}{\PYGZhy{}}\PYG{o}{\PYGZhy{}}\PYG{o}{\PYGZhy{}}\PYG{o}{\PYGZhy{}}\PYG{o}{\PYGZhy{}}\PYG{o}{\PYGZhy{}}\PYG{o}{\PYGZhy{}}\PYG{o}{\PYGZhy{}}\PYG{o}{\PYGZhy{}}\PYG{o}{\PYGZhy{}}\PYG{o}{\PYGZhy{}}\PYG{o}{\PYGZhy{}}\PYG{o}{\PYGZhy{}}\PYG{o}{\PYGZhy{}}\PYG{o}{\PYGZhy{}}\PYG{o}{\PYGZhy{}}\PYG{o}{\PYGZhy{}}\PYG{o}{\PYGZhy{}}\PYG{o}{\PYGZhy{}}\PYG{o}{\PYGZhy{}}\PYG{o}{\PYGZhy{}}\PYG{o}{\PYGZhy{}}\PYG{o}{\PYGZhy{}}\PYG{o}{\PYGZhy{}}\PYG{o}{\PYGZhy{}}\PYG{o}{\PYGZhy{}}\PYG{o}{\PYGZhy{}}\PYG{o}{\PYGZhy{}}\PYG{o}{\PYGZhy{}}\PYG{o}{\PYGZhy{}}\PYG{o}{\PYGZhy{}}\PYG{o}{\PYGZhy{}}\PYG{o}{\PYGZhy{}}
\PYG{n}{dt}\PYG{p}{:}         \PYG{p}{(}\PYG{n}{double}\PYG{p}{)}
            \PYG{p}{(}\PYG{l+m+mf}{0.01}\PYG{p}{)}\PYG{p}{[}\PYG{n}{ps}\PYG{p}{]} \PYG{n}{Time} \PYG{n}{step} \PYG{k}{for} \PYG{n}{integration}
\PYG{o}{\PYGZhy{}}\PYG{o}{\PYGZhy{}}\PYG{o}{\PYGZhy{}}\PYG{o}{\PYGZhy{}}\PYG{o}{\PYGZhy{}}\PYG{o}{\PYGZhy{}}\PYG{o}{\PYGZhy{}}\PYG{o}{\PYGZhy{}}\PYG{o}{\PYGZhy{}}\PYG{o}{\PYGZhy{}}\PYG{o}{\PYGZhy{}}\PYG{o}{\PYGZhy{}}\PYG{o}{\PYGZhy{}}\PYG{o}{\PYGZhy{}}\PYG{o}{\PYGZhy{}}\PYG{o}{\PYGZhy{}}\PYG{o}{\PYGZhy{}}\PYG{o}{\PYGZhy{}}\PYG{o}{\PYGZhy{}}\PYG{o}{\PYGZhy{}}\PYG{o}{\PYGZhy{}}\PYG{o}{\PYGZhy{}}\PYG{o}{\PYGZhy{}}\PYG{o}{\PYGZhy{}}\PYG{o}{\PYGZhy{}}\PYG{o}{\PYGZhy{}}\PYG{o}{\PYGZhy{}}\PYG{o}{\PYGZhy{}}\PYG{o}{\PYGZhy{}}\PYG{o}{\PYGZhy{}}\PYG{o}{\PYGZhy{}}\PYG{o}{\PYGZhy{}}\PYG{o}{\PYGZhy{}}\PYG{o}{\PYGZhy{}}\PYG{o}{\PYGZhy{}}\PYG{o}{\PYGZhy{}}\PYG{o}{\PYGZhy{}}\PYG{o}{\PYGZhy{}}\PYG{o}{\PYGZhy{}}\PYG{o}{\PYGZhy{}}\PYG{o}{\PYGZhy{}}\PYG{o}{\PYGZhy{}}\PYG{o}{\PYGZhy{}}\PYG{o}{\PYGZhy{}}\PYG{o}{\PYGZhy{}}\PYG{o}{\PYGZhy{}}\PYG{o}{\PYGZhy{}}\PYG{o}{\PYGZhy{}}\PYG{o}{\PYGZhy{}}\PYG{o}{\PYGZhy{}}\PYG{o}{\PYGZhy{}}\PYG{o}{\PYGZhy{}}\PYG{o}{\PYGZhy{}}\PYG{o}{\PYGZhy{}}\PYG{o}{\PYGZhy{}}\PYG{o}{\PYGZhy{}}\PYG{o}{\PYGZhy{}}\PYG{o}{\PYGZhy{}}\PYG{o}{\PYGZhy{}}\PYG{o}{\PYGZhy{}}\PYG{o}{\PYGZhy{}}\PYG{o}{\PYGZhy{}}\PYG{o}{\PYGZhy{}}\PYG{o}{\PYGZhy{}}\PYG{o}{\PYGZhy{}}\PYG{o}{\PYGZhy{}}\PYG{o}{\PYGZhy{}}\PYG{o}{\PYGZhy{}}\PYG{o}{\PYGZhy{}}\PYG{o}{\PYGZhy{}}\PYG{o}{\PYGZhy{}}\PYG{o}{\PYGZhy{}}\PYG{o}{\PYGZhy{}}\PYG{o}{\PYGZhy{}}\PYG{o}{\PYGZhy{}}\PYG{o}{\PYGZhy{}}\PYG{o}{\PYGZhy{}}\PYG{o}{\PYGZhy{}}\PYG{o}{\PYGZhy{}}\PYG{o}{\PYGZhy{}}\PYG{o}{\PYGZhy{}}\PYG{o}{\PYGZhy{}}\PYG{o}{\PYGZhy{}}\PYG{o}{\PYGZhy{}}
\PYG{n}{nstxout}\PYG{p}{:}    \PYG{p}{(}\PYG{n+nb}{int}\PYG{p}{)}
            \PYG{p}{(}\PYG{l+m+mi}{1}\PYG{p}{)} \PYG{p}{[}\PYG{n}{step}\PYG{p}{]} \PYG{n}{number} \PYG{n}{of} \PYG{n}{steps} \PYG{n}{that} \PYG{n}{elapse} \PYG{n}{between} \PYG{n}{writing} \PYG{n}{coordinates} \PYG{n}{to} \PYG{n}{output} \PYG{n}{trajectory} \PYG{n}{file}\PYG{p}{,}
                  \PYG{n}{the} \PYG{n}{last} \PYG{n}{coordinates} \PYG{n}{are} \PYG{n}{always} \PYG{n}{written}
\PYG{o}{\PYGZhy{}}\PYG{o}{\PYGZhy{}}\PYG{o}{\PYGZhy{}}\PYG{o}{\PYGZhy{}}\PYG{o}{\PYGZhy{}}\PYG{o}{\PYGZhy{}}\PYG{o}{\PYGZhy{}}\PYG{o}{\PYGZhy{}}\PYG{o}{\PYGZhy{}}\PYG{o}{\PYGZhy{}}\PYG{o}{\PYGZhy{}}\PYG{o}{\PYGZhy{}}\PYG{o}{\PYGZhy{}}\PYG{o}{\PYGZhy{}}\PYG{o}{\PYGZhy{}}\PYG{o}{\PYGZhy{}}\PYG{o}{\PYGZhy{}}\PYG{o}{\PYGZhy{}}\PYG{o}{\PYGZhy{}}\PYG{o}{\PYGZhy{}}\PYG{o}{\PYGZhy{}}\PYG{o}{\PYGZhy{}}\PYG{o}{\PYGZhy{}}\PYG{o}{\PYGZhy{}}\PYG{o}{\PYGZhy{}}\PYG{o}{\PYGZhy{}}\PYG{o}{\PYGZhy{}}\PYG{o}{\PYGZhy{}}\PYG{o}{\PYGZhy{}}\PYG{o}{\PYGZhy{}}\PYG{o}{\PYGZhy{}}\PYG{o}{\PYGZhy{}}\PYG{o}{\PYGZhy{}}\PYG{o}{\PYGZhy{}}\PYG{o}{\PYGZhy{}}\PYG{o}{\PYGZhy{}}\PYG{o}{\PYGZhy{}}\PYG{o}{\PYGZhy{}}\PYG{o}{\PYGZhy{}}\PYG{o}{\PYGZhy{}}\PYG{o}{\PYGZhy{}}\PYG{o}{\PYGZhy{}}\PYG{o}{\PYGZhy{}}\PYG{o}{\PYGZhy{}}\PYG{o}{\PYGZhy{}}\PYG{o}{\PYGZhy{}}\PYG{o}{\PYGZhy{}}\PYG{o}{\PYGZhy{}}\PYG{o}{\PYGZhy{}}\PYG{o}{\PYGZhy{}}\PYG{o}{\PYGZhy{}}\PYG{o}{\PYGZhy{}}\PYG{o}{\PYGZhy{}}\PYG{o}{\PYGZhy{}}\PYG{o}{\PYGZhy{}}\PYG{o}{\PYGZhy{}}\PYG{o}{\PYGZhy{}}\PYG{o}{\PYGZhy{}}\PYG{o}{\PYGZhy{}}\PYG{o}{\PYGZhy{}}\PYG{o}{\PYGZhy{}}\PYG{o}{\PYGZhy{}}\PYG{o}{\PYGZhy{}}\PYG{o}{\PYGZhy{}}\PYG{o}{\PYGZhy{}}\PYG{o}{\PYGZhy{}}\PYG{o}{\PYGZhy{}}\PYG{o}{\PYGZhy{}}\PYG{o}{\PYGZhy{}}\PYG{o}{\PYGZhy{}}\PYG{o}{\PYGZhy{}}\PYG{o}{\PYGZhy{}}\PYG{o}{\PYGZhy{}}\PYG{o}{\PYGZhy{}}\PYG{o}{\PYGZhy{}}\PYG{o}{\PYGZhy{}}\PYG{o}{\PYGZhy{}}\PYG{o}{\PYGZhy{}}\PYG{o}{\PYGZhy{}}\PYG{o}{\PYGZhy{}}\PYG{o}{\PYGZhy{}}\PYG{o}{\PYGZhy{}}\PYG{o}{\PYGZhy{}}\PYG{o}{\PYGZhy{}}
\PYG{n}{nstlog}\PYG{p}{:}     \PYG{p}{(}\PYG{n+nb}{int}\PYG{p}{)}
            \PYG{p}{(}\PYG{l+m+mi}{1}\PYG{p}{)} \PYG{n}{number} \PYG{n}{of} \PYG{n}{steps} \PYG{n}{that} \PYG{n}{elapse} \PYG{n}{between} \PYG{n}{writing} \PYG{n}{energies} \PYG{n}{to} \PYG{n}{the} \PYG{n}{log} \PYG{n}{file}
\PYG{o}{\PYGZhy{}}\PYG{o}{\PYGZhy{}}\PYG{o}{\PYGZhy{}}\PYG{o}{\PYGZhy{}}\PYG{o}{\PYGZhy{}}\PYG{o}{\PYGZhy{}}\PYG{o}{\PYGZhy{}}\PYG{o}{\PYGZhy{}}\PYG{o}{\PYGZhy{}}\PYG{o}{\PYGZhy{}}\PYG{o}{\PYGZhy{}}\PYG{o}{\PYGZhy{}}\PYG{o}{\PYGZhy{}}\PYG{o}{\PYGZhy{}}\PYG{o}{\PYGZhy{}}\PYG{o}{\PYGZhy{}}\PYG{o}{\PYGZhy{}}\PYG{o}{\PYGZhy{}}\PYG{o}{\PYGZhy{}}\PYG{o}{\PYGZhy{}}\PYG{o}{\PYGZhy{}}\PYG{o}{\PYGZhy{}}\PYG{o}{\PYGZhy{}}\PYG{o}{\PYGZhy{}}\PYG{o}{\PYGZhy{}}\PYG{o}{\PYGZhy{}}\PYG{o}{\PYGZhy{}}\PYG{o}{\PYGZhy{}}\PYG{o}{\PYGZhy{}}\PYG{o}{\PYGZhy{}}\PYG{o}{\PYGZhy{}}\PYG{o}{\PYGZhy{}}\PYG{o}{\PYGZhy{}}\PYG{o}{\PYGZhy{}}\PYG{o}{\PYGZhy{}}\PYG{o}{\PYGZhy{}}\PYG{o}{\PYGZhy{}}\PYG{o}{\PYGZhy{}}\PYG{o}{\PYGZhy{}}\PYG{o}{\PYGZhy{}}\PYG{o}{\PYGZhy{}}\PYG{o}{\PYGZhy{}}\PYG{o}{\PYGZhy{}}\PYG{o}{\PYGZhy{}}\PYG{o}{\PYGZhy{}}\PYG{o}{\PYGZhy{}}\PYG{o}{\PYGZhy{}}\PYG{o}{\PYGZhy{}}\PYG{o}{\PYGZhy{}}\PYG{o}{\PYGZhy{}}\PYG{o}{\PYGZhy{}}\PYG{o}{\PYGZhy{}}\PYG{o}{\PYGZhy{}}\PYG{o}{\PYGZhy{}}\PYG{o}{\PYGZhy{}}\PYG{o}{\PYGZhy{}}\PYG{o}{\PYGZhy{}}\PYG{o}{\PYGZhy{}}\PYG{o}{\PYGZhy{}}\PYG{o}{\PYGZhy{}}\PYG{o}{\PYGZhy{}}\PYG{o}{\PYGZhy{}}\PYG{o}{\PYGZhy{}}\PYG{o}{\PYGZhy{}}\PYG{o}{\PYGZhy{}}\PYG{o}{\PYGZhy{}}\PYG{o}{\PYGZhy{}}\PYG{o}{\PYGZhy{}}\PYG{o}{\PYGZhy{}}\PYG{o}{\PYGZhy{}}\PYG{o}{\PYGZhy{}}\PYG{o}{\PYGZhy{}}\PYG{o}{\PYGZhy{}}\PYG{o}{\PYGZhy{}}\PYG{o}{\PYGZhy{}}\PYG{o}{\PYGZhy{}}\PYG{o}{\PYGZhy{}}\PYG{o}{\PYGZhy{}}\PYG{o}{\PYGZhy{}}\PYG{o}{\PYGZhy{}}\PYG{o}{\PYGZhy{}}\PYG{o}{\PYGZhy{}}\PYG{o}{\PYGZhy{}}\PYG{o}{\PYGZhy{}}
\PYG{n}{nstcomm}\PYG{p}{:}    \PYG{p}{(}\PYG{n+nb}{int}\PYG{p}{)}
            \PYG{p}{(}\PYG{l+m+mi}{100}\PYG{p}{)} \PYG{n}{frequency} \PYG{k}{for} \PYG{n}{center} \PYG{n}{of} \PYG{n}{mass} \PYG{n}{motion} \PYG{n}{removal}
\end{sphinxVerbatim}


\section{Model parameter}
\label{\detokenize{usage/simulation_control:model-parameter}}
\sphinxAtStartPar
There are three models supported now: \sphinxtitleref{hps\_kr}, \sphinxtitleref{hps\_urry} and \sphinxtitleref{hps\_ss}.

\sphinxAtStartPar
\sphinxtitleref{hps\_kr} has parameters for a wide range of residues, i.e RNA, phosphorylation residues … but this model is less accurate

\begin{sphinxVerbatim}[commandchars=\\\{\}]
\PYG{n}{model}\PYG{p}{:}      \PYG{p}{(}\PYG{n}{string}\PYG{p}{)}
            \PYG{n}{hps\PYGZus{}kr}\PYG{p}{:} \PYG{n}{Kapcha}\PYG{o}{\PYGZhy{}}\PYG{n}{Rossy} \PYG{n}{hydropathy} \PYG{n}{scale}\PYG{p}{,} \PYG{n}{parameterize} \PYG{k+kn}{from} \PYG{n+nn}{OPLS}\PYG{o}{\PYGZhy{}}\PYG{n}{AA} \PYG{n}{forcefield}

            \PYG{n}{hps\PYGZus{}urry} \PYG{p}{(}\PYG{n}{default}\PYG{p}{)}\PYG{p}{:} \PYG{n}{Urry} \PYG{n}{hydropathy} \PYG{n}{scale}\PYG{p}{,} \PYG{n}{parameterize} \PYG{k+kn}{from} \PYG{n+nn}{experiment}

            \PYG{n}{hps\PYGZus{}ss}\PYG{p}{:} \PYG{n}{hps\PYGZus{}urry} \PYG{k}{with} \PYG{n}{bonded} \PYG{n}{potential} \PYG{p}{(}\PYG{n}{angle} \PYG{o+ow}{and} \PYG{n}{torsion}\PYG{p}{)}
\end{sphinxVerbatim}


\section{Temperature coupling}
\label{\detokenize{usage/simulation_control:temperature-coupling}}
\begin{sphinxVerbatim}[commandchars=\\\{\}]
\PYG{n}{tcoupl}\PYG{p}{:}     \PYG{p}{(}\PYG{n+nb}{bool}\PYG{p}{)}
            \PYG{n}{yes} \PYG{p}{(}\PYG{n}{default}\PYG{p}{)} \PYG{p}{:} \PYG{n}{The} \PYG{n}{only} \PYG{n}{available} \PYG{n}{option} \PYG{k}{for} \PYG{n}{now}\PYG{p}{,} \PYG{n}{we} \PYG{n}{don}\PYG{l+s+s1}{\PYGZsq{}}\PYG{l+s+s1}{t care about NVE ensemble.}
\PYG{o}{\PYGZhy{}}\PYG{o}{\PYGZhy{}}\PYG{o}{\PYGZhy{}}\PYG{o}{\PYGZhy{}}\PYG{o}{\PYGZhy{}}\PYG{o}{\PYGZhy{}}\PYG{o}{\PYGZhy{}}\PYG{o}{\PYGZhy{}}\PYG{o}{\PYGZhy{}}\PYG{o}{\PYGZhy{}}\PYG{o}{\PYGZhy{}}\PYG{o}{\PYGZhy{}}\PYG{o}{\PYGZhy{}}\PYG{o}{\PYGZhy{}}\PYG{o}{\PYGZhy{}}\PYG{o}{\PYGZhy{}}\PYG{o}{\PYGZhy{}}\PYG{o}{\PYGZhy{}}\PYG{o}{\PYGZhy{}}\PYG{o}{\PYGZhy{}}\PYG{o}{\PYGZhy{}}\PYG{o}{\PYGZhy{}}\PYG{o}{\PYGZhy{}}\PYG{o}{\PYGZhy{}}\PYG{o}{\PYGZhy{}}\PYG{o}{\PYGZhy{}}\PYG{o}{\PYGZhy{}}\PYG{o}{\PYGZhy{}}\PYG{o}{\PYGZhy{}}\PYG{o}{\PYGZhy{}}\PYG{o}{\PYGZhy{}}\PYG{o}{\PYGZhy{}}\PYG{o}{\PYGZhy{}}\PYG{o}{\PYGZhy{}}\PYG{o}{\PYGZhy{}}\PYG{o}{\PYGZhy{}}\PYG{o}{\PYGZhy{}}\PYG{o}{\PYGZhy{}}\PYG{o}{\PYGZhy{}}\PYG{o}{\PYGZhy{}}\PYG{o}{\PYGZhy{}}\PYG{o}{\PYGZhy{}}\PYG{o}{\PYGZhy{}}\PYG{o}{\PYGZhy{}}\PYG{o}{\PYGZhy{}}\PYG{o}{\PYGZhy{}}\PYG{o}{\PYGZhy{}}\PYG{o}{\PYGZhy{}}\PYG{o}{\PYGZhy{}}\PYG{o}{\PYGZhy{}}\PYG{o}{\PYGZhy{}}\PYG{o}{\PYGZhy{}}\PYG{o}{\PYGZhy{}}\PYG{o}{\PYGZhy{}}\PYG{o}{\PYGZhy{}}\PYG{o}{\PYGZhy{}}\PYG{o}{\PYGZhy{}}\PYG{o}{\PYGZhy{}}\PYG{o}{\PYGZhy{}}\PYG{o}{\PYGZhy{}}\PYG{o}{\PYGZhy{}}\PYG{o}{\PYGZhy{}}\PYG{o}{\PYGZhy{}}\PYG{o}{\PYGZhy{}}\PYG{o}{\PYGZhy{}}\PYG{o}{\PYGZhy{}}\PYG{o}{\PYGZhy{}}\PYG{o}{\PYGZhy{}}\PYG{o}{\PYGZhy{}}\PYG{o}{\PYGZhy{}}\PYG{o}{\PYGZhy{}}\PYG{o}{\PYGZhy{}}\PYG{o}{\PYGZhy{}}\PYG{o}{\PYGZhy{}}\PYG{o}{\PYGZhy{}}\PYG{o}{\PYGZhy{}}\PYG{o}{\PYGZhy{}}\PYG{o}{\PYGZhy{}}\PYG{o}{\PYGZhy{}}\PYG{o}{\PYGZhy{}}\PYG{o}{\PYGZhy{}}\PYG{o}{\PYGZhy{}}\PYG{o}{\PYGZhy{}}\PYG{o}{\PYGZhy{}}
\PYG{n}{ref\PYGZus{}t}\PYG{p}{:}      \PYG{p}{(}\PYG{n}{double}\PYG{p}{)}
            \PYG{p}{(}\PYG{l+m+mi}{300}\PYG{p}{)} \PYG{p}{[}\PYG{n}{K}\PYG{p}{]} \PYG{p}{:} \PYG{n}{Reference} \PYG{n}{temperature} \PYG{o+ow}{in} \PYG{n}{unit} \PYG{n}{of} \PYG{n}{Kelvin}
\PYG{o}{\PYGZhy{}}\PYG{o}{\PYGZhy{}}\PYG{o}{\PYGZhy{}}\PYG{o}{\PYGZhy{}}\PYG{o}{\PYGZhy{}}\PYG{o}{\PYGZhy{}}\PYG{o}{\PYGZhy{}}\PYG{o}{\PYGZhy{}}\PYG{o}{\PYGZhy{}}\PYG{o}{\PYGZhy{}}\PYG{o}{\PYGZhy{}}\PYG{o}{\PYGZhy{}}\PYG{o}{\PYGZhy{}}\PYG{o}{\PYGZhy{}}\PYG{o}{\PYGZhy{}}\PYG{o}{\PYGZhy{}}\PYG{o}{\PYGZhy{}}\PYG{o}{\PYGZhy{}}\PYG{o}{\PYGZhy{}}\PYG{o}{\PYGZhy{}}\PYG{o}{\PYGZhy{}}\PYG{o}{\PYGZhy{}}\PYG{o}{\PYGZhy{}}\PYG{o}{\PYGZhy{}}\PYG{o}{\PYGZhy{}}\PYG{o}{\PYGZhy{}}\PYG{o}{\PYGZhy{}}\PYG{o}{\PYGZhy{}}\PYG{o}{\PYGZhy{}}\PYG{o}{\PYGZhy{}}\PYG{o}{\PYGZhy{}}\PYG{o}{\PYGZhy{}}\PYG{o}{\PYGZhy{}}\PYG{o}{\PYGZhy{}}\PYG{o}{\PYGZhy{}}\PYG{o}{\PYGZhy{}}\PYG{o}{\PYGZhy{}}\PYG{o}{\PYGZhy{}}\PYG{o}{\PYGZhy{}}\PYG{o}{\PYGZhy{}}\PYG{o}{\PYGZhy{}}\PYG{o}{\PYGZhy{}}\PYG{o}{\PYGZhy{}}\PYG{o}{\PYGZhy{}}\PYG{o}{\PYGZhy{}}\PYG{o}{\PYGZhy{}}\PYG{o}{\PYGZhy{}}\PYG{o}{\PYGZhy{}}\PYG{o}{\PYGZhy{}}\PYG{o}{\PYGZhy{}}\PYG{o}{\PYGZhy{}}\PYG{o}{\PYGZhy{}}\PYG{o}{\PYGZhy{}}\PYG{o}{\PYGZhy{}}\PYG{o}{\PYGZhy{}}\PYG{o}{\PYGZhy{}}\PYG{o}{\PYGZhy{}}\PYG{o}{\PYGZhy{}}\PYG{o}{\PYGZhy{}}\PYG{o}{\PYGZhy{}}\PYG{o}{\PYGZhy{}}\PYG{o}{\PYGZhy{}}\PYG{o}{\PYGZhy{}}\PYG{o}{\PYGZhy{}}\PYG{o}{\PYGZhy{}}\PYG{o}{\PYGZhy{}}\PYG{o}{\PYGZhy{}}\PYG{o}{\PYGZhy{}}\PYG{o}{\PYGZhy{}}\PYG{o}{\PYGZhy{}}\PYG{o}{\PYGZhy{}}\PYG{o}{\PYGZhy{}}\PYG{o}{\PYGZhy{}}\PYG{o}{\PYGZhy{}}\PYG{o}{\PYGZhy{}}\PYG{o}{\PYGZhy{}}\PYG{o}{\PYGZhy{}}\PYG{o}{\PYGZhy{}}\PYG{o}{\PYGZhy{}}\PYG{o}{\PYGZhy{}}\PYG{o}{\PYGZhy{}}\PYG{o}{\PYGZhy{}}\PYG{o}{\PYGZhy{}}\PYG{o}{\PYGZhy{}}
\PYG{n}{tau\PYGZus{}t}\PYG{p}{:}      \PYG{p}{(}\PYG{n}{double}\PYG{p}{)}
            \PYG{p}{[}\PYG{n}{ps}\PYG{o}{\PYGZca{}}\PYG{o}{\PYGZhy{}}\PYG{l+m+mi}{1}\PYG{p}{]} \PYG{p}{:} \PYG{n}{The} \PYG{n}{friction} \PYG{n}{coefficient} \PYG{n}{which} \PYG{n}{couples} \PYG{n}{the} \PYG{n}{system} \PYG{n}{to} \PYG{n}{the} \PYG{n}{heat} \PYG{n}{bath} \PYG{p}{(}\PYG{o+ow}{in} \PYG{n}{inverse} \PYG{n}{picoseconds}\PYG{p}{)}
\end{sphinxVerbatim}


\section{Pressure coupling}
\label{\detokenize{usage/simulation_control:pressure-coupling}}
\begin{sphinxVerbatim}[commandchars=\\\{\}]
\PYG{n}{pcoupl}      \PYG{p}{(}\PYG{n+nb}{bool}\PYG{p}{)}
            \PYG{n}{yes} \PYG{p}{:} \PYG{n}{Using} \PYG{n}{pressure} \PYG{n}{coupling}

            \PYG{n}{no} \PYG{p}{(}\PYG{n}{default}\PYG{p}{)} \PYG{p}{:} \PYG{n}{Run} \PYG{n}{on} \PYG{n}{NVT} \PYG{n}{ensemble} \PYG{n}{only}
\PYG{o}{\PYGZhy{}}\PYG{o}{\PYGZhy{}}\PYG{o}{\PYGZhy{}}\PYG{o}{\PYGZhy{}}\PYG{o}{\PYGZhy{}}\PYG{o}{\PYGZhy{}}\PYG{o}{\PYGZhy{}}\PYG{o}{\PYGZhy{}}\PYG{o}{\PYGZhy{}}\PYG{o}{\PYGZhy{}}\PYG{o}{\PYGZhy{}}\PYG{o}{\PYGZhy{}}\PYG{o}{\PYGZhy{}}\PYG{o}{\PYGZhy{}}\PYG{o}{\PYGZhy{}}\PYG{o}{\PYGZhy{}}\PYG{o}{\PYGZhy{}}\PYG{o}{\PYGZhy{}}\PYG{o}{\PYGZhy{}}\PYG{o}{\PYGZhy{}}\PYG{o}{\PYGZhy{}}\PYG{o}{\PYGZhy{}}\PYG{o}{\PYGZhy{}}\PYG{o}{\PYGZhy{}}\PYG{o}{\PYGZhy{}}\PYG{o}{\PYGZhy{}}\PYG{o}{\PYGZhy{}}\PYG{o}{\PYGZhy{}}\PYG{o}{\PYGZhy{}}\PYG{o}{\PYGZhy{}}\PYG{o}{\PYGZhy{}}\PYG{o}{\PYGZhy{}}\PYG{o}{\PYGZhy{}}\PYG{o}{\PYGZhy{}}\PYG{o}{\PYGZhy{}}\PYG{o}{\PYGZhy{}}\PYG{o}{\PYGZhy{}}\PYG{o}{\PYGZhy{}}\PYG{o}{\PYGZhy{}}\PYG{o}{\PYGZhy{}}\PYG{o}{\PYGZhy{}}\PYG{o}{\PYGZhy{}}\PYG{o}{\PYGZhy{}}\PYG{o}{\PYGZhy{}}\PYG{o}{\PYGZhy{}}\PYG{o}{\PYGZhy{}}\PYG{o}{\PYGZhy{}}\PYG{o}{\PYGZhy{}}\PYG{o}{\PYGZhy{}}\PYG{o}{\PYGZhy{}}\PYG{o}{\PYGZhy{}}\PYG{o}{\PYGZhy{}}\PYG{o}{\PYGZhy{}}\PYG{o}{\PYGZhy{}}\PYG{o}{\PYGZhy{}}\PYG{o}{\PYGZhy{}}\PYG{o}{\PYGZhy{}}\PYG{o}{\PYGZhy{}}\PYG{o}{\PYGZhy{}}\PYG{o}{\PYGZhy{}}\PYG{o}{\PYGZhy{}}\PYG{o}{\PYGZhy{}}\PYG{o}{\PYGZhy{}}\PYG{o}{\PYGZhy{}}\PYG{o}{\PYGZhy{}}\PYG{o}{\PYGZhy{}}\PYG{o}{\PYGZhy{}}\PYG{o}{\PYGZhy{}}\PYG{o}{\PYGZhy{}}\PYG{o}{\PYGZhy{}}\PYG{o}{\PYGZhy{}}\PYG{o}{\PYGZhy{}}\PYG{o}{\PYGZhy{}}\PYG{o}{\PYGZhy{}}\PYG{o}{\PYGZhy{}}\PYG{o}{\PYGZhy{}}\PYG{o}{\PYGZhy{}}\PYG{o}{\PYGZhy{}}\PYG{o}{\PYGZhy{}}\PYG{o}{\PYGZhy{}}\PYG{o}{\PYGZhy{}}\PYG{o}{\PYGZhy{}}\PYG{o}{\PYGZhy{}}\PYG{o}{\PYGZhy{}}
\PYG{n}{ref\PYGZus{}p}       \PYG{p}{(}\PYG{n}{double}\PYG{p}{)}
             \PYG{p}{(}\PYG{l+m+mi}{1}\PYG{p}{)} \PYG{p}{[}\PYG{n}{bar}\PYG{p}{]} \PYG{n}{The} \PYG{n}{default} \PYG{n}{pressure} \PYG{n}{acting} \PYG{n}{on} \PYG{n}{the} \PYG{n}{system}\PYG{o}{.}
\PYG{o}{\PYGZhy{}}\PYG{o}{\PYGZhy{}}\PYG{o}{\PYGZhy{}}\PYG{o}{\PYGZhy{}}\PYG{o}{\PYGZhy{}}\PYG{o}{\PYGZhy{}}\PYG{o}{\PYGZhy{}}\PYG{o}{\PYGZhy{}}\PYG{o}{\PYGZhy{}}\PYG{o}{\PYGZhy{}}\PYG{o}{\PYGZhy{}}\PYG{o}{\PYGZhy{}}\PYG{o}{\PYGZhy{}}\PYG{o}{\PYGZhy{}}\PYG{o}{\PYGZhy{}}\PYG{o}{\PYGZhy{}}\PYG{o}{\PYGZhy{}}\PYG{o}{\PYGZhy{}}\PYG{o}{\PYGZhy{}}\PYG{o}{\PYGZhy{}}\PYG{o}{\PYGZhy{}}\PYG{o}{\PYGZhy{}}\PYG{o}{\PYGZhy{}}\PYG{o}{\PYGZhy{}}\PYG{o}{\PYGZhy{}}\PYG{o}{\PYGZhy{}}\PYG{o}{\PYGZhy{}}\PYG{o}{\PYGZhy{}}\PYG{o}{\PYGZhy{}}\PYG{o}{\PYGZhy{}}\PYG{o}{\PYGZhy{}}\PYG{o}{\PYGZhy{}}\PYG{o}{\PYGZhy{}}\PYG{o}{\PYGZhy{}}\PYG{o}{\PYGZhy{}}\PYG{o}{\PYGZhy{}}\PYG{o}{\PYGZhy{}}\PYG{o}{\PYGZhy{}}\PYG{o}{\PYGZhy{}}\PYG{o}{\PYGZhy{}}\PYG{o}{\PYGZhy{}}\PYG{o}{\PYGZhy{}}\PYG{o}{\PYGZhy{}}\PYG{o}{\PYGZhy{}}\PYG{o}{\PYGZhy{}}\PYG{o}{\PYGZhy{}}\PYG{o}{\PYGZhy{}}\PYG{o}{\PYGZhy{}}\PYG{o}{\PYGZhy{}}\PYG{o}{\PYGZhy{}}\PYG{o}{\PYGZhy{}}\PYG{o}{\PYGZhy{}}\PYG{o}{\PYGZhy{}}\PYG{o}{\PYGZhy{}}\PYG{o}{\PYGZhy{}}\PYG{o}{\PYGZhy{}}\PYG{o}{\PYGZhy{}}\PYG{o}{\PYGZhy{}}\PYG{o}{\PYGZhy{}}\PYG{o}{\PYGZhy{}}\PYG{o}{\PYGZhy{}}\PYG{o}{\PYGZhy{}}\PYG{o}{\PYGZhy{}}\PYG{o}{\PYGZhy{}}\PYG{o}{\PYGZhy{}}\PYG{o}{\PYGZhy{}}\PYG{o}{\PYGZhy{}}\PYG{o}{\PYGZhy{}}\PYG{o}{\PYGZhy{}}\PYG{o}{\PYGZhy{}}\PYG{o}{\PYGZhy{}}\PYG{o}{\PYGZhy{}}\PYG{o}{\PYGZhy{}}\PYG{o}{\PYGZhy{}}\PYG{o}{\PYGZhy{}}\PYG{o}{\PYGZhy{}}\PYG{o}{\PYGZhy{}}\PYG{o}{\PYGZhy{}}\PYG{o}{\PYGZhy{}}\PYG{o}{\PYGZhy{}}\PYG{o}{\PYGZhy{}}\PYG{o}{\PYGZhy{}}\PYG{o}{\PYGZhy{}}\PYG{o}{\PYGZhy{}}
\PYG{n}{frequency\PYGZus{}p} \PYG{p}{(}\PYG{n+nb}{int}\PYG{p}{)}
            \PYG{p}{(}\PYG{l+m+mi}{25}\PYG{p}{)} \PYG{p}{[}\PYG{n}{steps}\PYG{p}{]} \PYG{n}{the} \PYG{n}{frequency} \PYG{n}{at} \PYG{n}{which} \PYG{n}{Monte} \PYG{n}{Carlo} \PYG{n}{pressure} \PYG{n}{changes} \PYG{n}{should} \PYG{n}{be} \PYG{n}{attempted}
\end{sphinxVerbatim}


\section{Periodic boundary condition:}
\label{\detokenize{usage/simulation_control:periodic-boundary-condition}}
\sphinxAtStartPar
if pcoupl is yes then pbc must be yes.

\begin{sphinxVerbatim}[commandchars=\\\{\}]
pbc         (bool)
            yes : Using periodic boundary condition.
                    If this option is chosen, then it will affect to non\PYGZhy{}bonded forces in the system,
                    and the coordinate writen in PDB and DCD file as well. No worries since I have handled these.

            no (default) : Without periodic boundary condition.
\PYGZhy{}\PYGZhy{}\PYGZhy{}\PYGZhy{}\PYGZhy{}\PYGZhy{}\PYGZhy{}\PYGZhy{}\PYGZhy{}\PYGZhy{}\PYGZhy{}\PYGZhy{}\PYGZhy{}\PYGZhy{}\PYGZhy{}\PYGZhy{}\PYGZhy{}\PYGZhy{}\PYGZhy{}\PYGZhy{}\PYGZhy{}\PYGZhy{}\PYGZhy{}\PYGZhy{}\PYGZhy{}\PYGZhy{}\PYGZhy{}\PYGZhy{}\PYGZhy{}\PYGZhy{}\PYGZhy{}\PYGZhy{}\PYGZhy{}\PYGZhy{}\PYGZhy{}\PYGZhy{}\PYGZhy{}\PYGZhy{}\PYGZhy{}\PYGZhy{}\PYGZhy{}\PYGZhy{}\PYGZhy{}\PYGZhy{}\PYGZhy{}\PYGZhy{}\PYGZhy{}\PYGZhy{}\PYGZhy{}\PYGZhy{}\PYGZhy{}\PYGZhy{}\PYGZhy{}\PYGZhy{}\PYGZhy{}\PYGZhy{}\PYGZhy{}\PYGZhy{}\PYGZhy{}\PYGZhy{}\PYGZhy{}\PYGZhy{}\PYGZhy{}\PYGZhy{}\PYGZhy{}\PYGZhy{}\PYGZhy{}\PYGZhy{}\PYGZhy{}\PYGZhy{}\PYGZhy{}\PYGZhy{}\PYGZhy{}\PYGZhy{}\PYGZhy{}\PYGZhy{}\PYGZhy{}\PYGZhy{}\PYGZhy{}\PYGZhy{}\PYGZhy{}\PYGZhy{}\PYGZhy{}\PYGZhy{}
box\PYGZus{}dimension   (float or list of float)
            [nm] An example of box dimension:
            If you want a cubic box of 30x30x30 nm\PYGZca{}3, put: 30 or [30, 30, 30]
            If you want a rectangular box? Put:  [30, 30, 60]
\end{sphinxVerbatim}


\section{File input/output}
\label{\detokenize{usage/simulation_control:file-input-output}}
\begin{sphinxVerbatim}[commandchars=\\\{\}]
\PYG{n}{protein\PYGZus{}code}    \PYG{p}{(}\PYG{n}{string}\PYG{p}{)}
                \PYG{n}{String} \PYG{k}{for} \PYG{n}{output} \PYG{n}{prefix}\PYG{p}{,} \PYG{n}{i}\PYG{o}{.}\PYG{n}{e} \PYG{p}{\PYGZob{}}\PYG{n}{protein\PYGZus{}code}\PYG{p}{\PYGZcb{}}\PYG{o}{.}\PYG{n}{dcd}\PYG{p}{,} \PYG{p}{\PYGZob{}}\PYG{n}{protein\PYGZus{}code}\PYG{p}{\PYGZcb{}}\PYG{o}{.}\PYG{n}{log}
\PYG{o}{\PYGZhy{}}\PYG{o}{\PYGZhy{}}\PYG{o}{\PYGZhy{}}\PYG{o}{\PYGZhy{}}\PYG{o}{\PYGZhy{}}\PYG{o}{\PYGZhy{}}\PYG{o}{\PYGZhy{}}\PYG{o}{\PYGZhy{}}\PYG{o}{\PYGZhy{}}\PYG{o}{\PYGZhy{}}\PYG{o}{\PYGZhy{}}\PYG{o}{\PYGZhy{}}\PYG{o}{\PYGZhy{}}\PYG{o}{\PYGZhy{}}\PYG{o}{\PYGZhy{}}\PYG{o}{\PYGZhy{}}\PYG{o}{\PYGZhy{}}\PYG{o}{\PYGZhy{}}\PYG{o}{\PYGZhy{}}\PYG{o}{\PYGZhy{}}\PYG{o}{\PYGZhy{}}\PYG{o}{\PYGZhy{}}\PYG{o}{\PYGZhy{}}\PYG{o}{\PYGZhy{}}\PYG{o}{\PYGZhy{}}\PYG{o}{\PYGZhy{}}\PYG{o}{\PYGZhy{}}\PYG{o}{\PYGZhy{}}\PYG{o}{\PYGZhy{}}\PYG{o}{\PYGZhy{}}\PYG{o}{\PYGZhy{}}\PYG{o}{\PYGZhy{}}\PYG{o}{\PYGZhy{}}\PYG{o}{\PYGZhy{}}\PYG{o}{\PYGZhy{}}\PYG{o}{\PYGZhy{}}\PYG{o}{\PYGZhy{}}\PYG{o}{\PYGZhy{}}\PYG{o}{\PYGZhy{}}\PYG{o}{\PYGZhy{}}\PYG{o}{\PYGZhy{}}\PYG{o}{\PYGZhy{}}\PYG{o}{\PYGZhy{}}\PYG{o}{\PYGZhy{}}\PYG{o}{\PYGZhy{}}\PYG{o}{\PYGZhy{}}\PYG{o}{\PYGZhy{}}\PYG{o}{\PYGZhy{}}\PYG{o}{\PYGZhy{}}\PYG{o}{\PYGZhy{}}\PYG{o}{\PYGZhy{}}\PYG{o}{\PYGZhy{}}\PYG{o}{\PYGZhy{}}\PYG{o}{\PYGZhy{}}\PYG{o}{\PYGZhy{}}\PYG{o}{\PYGZhy{}}\PYG{o}{\PYGZhy{}}\PYG{o}{\PYGZhy{}}\PYG{o}{\PYGZhy{}}\PYG{o}{\PYGZhy{}}\PYG{o}{\PYGZhy{}}\PYG{o}{\PYGZhy{}}\PYG{o}{\PYGZhy{}}\PYG{o}{\PYGZhy{}}\PYG{o}{\PYGZhy{}}\PYG{o}{\PYGZhy{}}\PYG{o}{\PYGZhy{}}\PYG{o}{\PYGZhy{}}\PYG{o}{\PYGZhy{}}\PYG{o}{\PYGZhy{}}\PYG{o}{\PYGZhy{}}\PYG{o}{\PYGZhy{}}\PYG{o}{\PYGZhy{}}\PYG{o}{\PYGZhy{}}\PYG{o}{\PYGZhy{}}\PYG{o}{\PYGZhy{}}\PYG{o}{\PYGZhy{}}\PYG{o}{\PYGZhy{}}\PYG{o}{\PYGZhy{}}\PYG{o}{\PYGZhy{}}\PYG{o}{\PYGZhy{}}\PYG{o}{\PYGZhy{}}\PYG{o}{\PYGZhy{}}\PYG{o}{\PYGZhy{}}
\PYG{n}{pdb\PYGZus{}file}        \PYG{p}{(}\PYG{n}{string}\PYG{p}{)}
                \PYG{p}{[}\PYG{o}{.}\PYG{n}{pdb}\PYG{p}{,} \PYG{o}{.}\PYG{n}{cif}\PYG{p}{]} \PYG{n}{Input} \PYG{n}{structure} \PYG{k}{for} \PYG{n}{loading} \PYG{n}{topology} \PYG{o+ow}{and} \PYG{n}{initial} \PYG{n}{coordinate}
\PYG{o}{\PYGZhy{}}\PYG{o}{\PYGZhy{}}\PYG{o}{\PYGZhy{}}\PYG{o}{\PYGZhy{}}\PYG{o}{\PYGZhy{}}\PYG{o}{\PYGZhy{}}\PYG{o}{\PYGZhy{}}\PYG{o}{\PYGZhy{}}\PYG{o}{\PYGZhy{}}\PYG{o}{\PYGZhy{}}\PYG{o}{\PYGZhy{}}\PYG{o}{\PYGZhy{}}\PYG{o}{\PYGZhy{}}\PYG{o}{\PYGZhy{}}\PYG{o}{\PYGZhy{}}\PYG{o}{\PYGZhy{}}\PYG{o}{\PYGZhy{}}\PYG{o}{\PYGZhy{}}\PYG{o}{\PYGZhy{}}\PYG{o}{\PYGZhy{}}\PYG{o}{\PYGZhy{}}\PYG{o}{\PYGZhy{}}\PYG{o}{\PYGZhy{}}\PYG{o}{\PYGZhy{}}\PYG{o}{\PYGZhy{}}\PYG{o}{\PYGZhy{}}\PYG{o}{\PYGZhy{}}\PYG{o}{\PYGZhy{}}\PYG{o}{\PYGZhy{}}\PYG{o}{\PYGZhy{}}\PYG{o}{\PYGZhy{}}\PYG{o}{\PYGZhy{}}\PYG{o}{\PYGZhy{}}\PYG{o}{\PYGZhy{}}\PYG{o}{\PYGZhy{}}\PYG{o}{\PYGZhy{}}\PYG{o}{\PYGZhy{}}\PYG{o}{\PYGZhy{}}\PYG{o}{\PYGZhy{}}\PYG{o}{\PYGZhy{}}\PYG{o}{\PYGZhy{}}\PYG{o}{\PYGZhy{}}\PYG{o}{\PYGZhy{}}\PYG{o}{\PYGZhy{}}\PYG{o}{\PYGZhy{}}\PYG{o}{\PYGZhy{}}\PYG{o}{\PYGZhy{}}\PYG{o}{\PYGZhy{}}\PYG{o}{\PYGZhy{}}\PYG{o}{\PYGZhy{}}\PYG{o}{\PYGZhy{}}\PYG{o}{\PYGZhy{}}\PYG{o}{\PYGZhy{}}\PYG{o}{\PYGZhy{}}\PYG{o}{\PYGZhy{}}\PYG{o}{\PYGZhy{}}\PYG{o}{\PYGZhy{}}\PYG{o}{\PYGZhy{}}\PYG{o}{\PYGZhy{}}\PYG{o}{\PYGZhy{}}\PYG{o}{\PYGZhy{}}\PYG{o}{\PYGZhy{}}\PYG{o}{\PYGZhy{}}\PYG{o}{\PYGZhy{}}\PYG{o}{\PYGZhy{}}\PYG{o}{\PYGZhy{}}\PYG{o}{\PYGZhy{}}\PYG{o}{\PYGZhy{}}\PYG{o}{\PYGZhy{}}\PYG{o}{\PYGZhy{}}\PYG{o}{\PYGZhy{}}\PYG{o}{\PYGZhy{}}\PYG{o}{\PYGZhy{}}\PYG{o}{\PYGZhy{}}\PYG{o}{\PYGZhy{}}\PYG{o}{\PYGZhy{}}\PYG{o}{\PYGZhy{}}\PYG{o}{\PYGZhy{}}\PYG{o}{\PYGZhy{}}\PYG{o}{\PYGZhy{}}\PYG{o}{\PYGZhy{}}\PYG{o}{\PYGZhy{}}\PYG{o}{\PYGZhy{}}\PYG{o}{\PYGZhy{}}
\PYG{n}{checkpoint}      \PYG{p}{(}\PYG{n}{string}\PYG{p}{)}
                \PYG{p}{[}\PYG{o}{.}\PYG{n}{chk}\PYG{p}{]} \PYG{n}{Checkpoint} \PYG{n}{file} \PYG{n}{name}\PYG{p}{,} \PYG{n}{here} \PYG{n}{I} \PYG{n}{ask} \PYG{n}{you} \PYG{n}{to} \PYG{n}{provide} \PYG{n}{it} \PYG{n}{explicitly} \PYG{n}{since}
                        \PYG{n}{because} \PYG{n}{checkpoint} \PYG{n}{can} \PYG{n}{be} \PYG{n}{used} \PYG{n}{to} \PYG{n}{load} \PYG{n}{state} \PYG{o+ow}{or} \PYG{n}{save} \PYG{n}{state}\PYG{o}{.}
                        \PYG{o+ow}{in} \PYG{n}{case} \PYG{k}{if} \PYG{n}{you} \PYG{n}{restart} \PYG{n}{simulation} \PYG{k}{with} \PYG{n}{different} \PYG{n}{name}\PYG{p}{,} \PYG{n}{you} \PYG{n}{have} \PYG{n}{to} \PYG{n}{provide} \PYG{n}{it}\PYG{o}{.}
\end{sphinxVerbatim}


\section{Simulation platform}
\label{\detokenize{usage/simulation_control:simulation-platform}}
\sphinxAtStartPar
Simulation can be run on CPU with number of threads is control by \sphinxtitleref{ppn} or using GPU.
If \sphinxtitleref{device=CPU} then ppn need to be specify, otherwise simulation will run on 1 core

\begin{sphinxVerbatim}[commandchars=\\\{\}]
\PYG{n}{device}          \PYG{p}{(}\PYG{n}{string}\PYG{p}{)}
                \PYG{n}{GPU} \PYG{p}{:} \PYG{n}{Use} \PYG{n}{gpu} \PYG{n}{to} \PYG{n}{run} \PYG{n}{simulation}

                \PYG{n}{CPU} \PYG{p}{(}\PYG{n}{default}\PYG{p}{)} \PYG{p}{:} \PYG{n}{use} \PYG{n}{cpu} \PYG{n}{to} \PYG{n}{run} \PYG{n}{simulation}\PYG{p}{,} \PYG{k}{if} \PYG{n}{you} \PYG{n}{specify} \PYG{n}{cpu}\PYG{p}{,} \PYG{n}{you} \PYG{n}{should} \PYG{n}{modify} \PYG{n}{ppn} \PYG{n}{option}\PYG{p}{,} \PYG{n}{it} \PYG{n}{control}
                        \PYG{n}{how} \PYG{n}{many} \PYG{n}{cores} \PYG{n}{will} \PYG{n}{be} \PYG{n}{used} \PYG{n}{to} \PYG{n}{run} \PYG{n}{simulation}\PYG{p}{,} \PYG{k}{if} \PYG{o+ow}{not}\PYG{p}{,} \PYG{n}{default} \PYG{o+ow}{is} \PYG{l+m+mf}{1.}
\PYG{o}{\PYGZhy{}}\PYG{o}{\PYGZhy{}}\PYG{o}{\PYGZhy{}}\PYG{o}{\PYGZhy{}}\PYG{o}{\PYGZhy{}}\PYG{o}{\PYGZhy{}}\PYG{o}{\PYGZhy{}}\PYG{o}{\PYGZhy{}}\PYG{o}{\PYGZhy{}}\PYG{o}{\PYGZhy{}}\PYG{o}{\PYGZhy{}}\PYG{o}{\PYGZhy{}}\PYG{o}{\PYGZhy{}}\PYG{o}{\PYGZhy{}}\PYG{o}{\PYGZhy{}}\PYG{o}{\PYGZhy{}}\PYG{o}{\PYGZhy{}}\PYG{o}{\PYGZhy{}}\PYG{o}{\PYGZhy{}}\PYG{o}{\PYGZhy{}}\PYG{o}{\PYGZhy{}}\PYG{o}{\PYGZhy{}}\PYG{o}{\PYGZhy{}}\PYG{o}{\PYGZhy{}}\PYG{o}{\PYGZhy{}}\PYG{o}{\PYGZhy{}}\PYG{o}{\PYGZhy{}}\PYG{o}{\PYGZhy{}}\PYG{o}{\PYGZhy{}}\PYG{o}{\PYGZhy{}}\PYG{o}{\PYGZhy{}}\PYG{o}{\PYGZhy{}}\PYG{o}{\PYGZhy{}}\PYG{o}{\PYGZhy{}}\PYG{o}{\PYGZhy{}}\PYG{o}{\PYGZhy{}}\PYG{o}{\PYGZhy{}}\PYG{o}{\PYGZhy{}}\PYG{o}{\PYGZhy{}}\PYG{o}{\PYGZhy{}}\PYG{o}{\PYGZhy{}}\PYG{o}{\PYGZhy{}}\PYG{o}{\PYGZhy{}}\PYG{o}{\PYGZhy{}}\PYG{o}{\PYGZhy{}}\PYG{o}{\PYGZhy{}}\PYG{o}{\PYGZhy{}}\PYG{o}{\PYGZhy{}}\PYG{o}{\PYGZhy{}}\PYG{o}{\PYGZhy{}}\PYG{o}{\PYGZhy{}}\PYG{o}{\PYGZhy{}}\PYG{o}{\PYGZhy{}}\PYG{o}{\PYGZhy{}}\PYG{o}{\PYGZhy{}}\PYG{o}{\PYGZhy{}}\PYG{o}{\PYGZhy{}}\PYG{o}{\PYGZhy{}}\PYG{o}{\PYGZhy{}}\PYG{o}{\PYGZhy{}}\PYG{o}{\PYGZhy{}}\PYG{o}{\PYGZhy{}}\PYG{o}{\PYGZhy{}}\PYG{o}{\PYGZhy{}}\PYG{o}{\PYGZhy{}}\PYG{o}{\PYGZhy{}}\PYG{o}{\PYGZhy{}}\PYG{o}{\PYGZhy{}}\PYG{o}{\PYGZhy{}}\PYG{o}{\PYGZhy{}}\PYG{o}{\PYGZhy{}}\PYG{o}{\PYGZhy{}}\PYG{o}{\PYGZhy{}}\PYG{o}{\PYGZhy{}}\PYG{o}{\PYGZhy{}}\PYG{o}{\PYGZhy{}}\PYG{o}{\PYGZhy{}}\PYG{o}{\PYGZhy{}}\PYG{o}{\PYGZhy{}}\PYG{o}{\PYGZhy{}}\PYG{o}{\PYGZhy{}}\PYG{o}{\PYGZhy{}}\PYG{o}{\PYGZhy{}}\PYG{o}{\PYGZhy{}}
\PYG{n}{ppn}             \PYG{p}{(}\PYG{n+nb}{int}\PYG{p}{)}
                \PYG{p}{(}\PYG{l+m+mi}{1}\PYG{p}{)} \PYG{p}{[}\PYG{n}{threads}\PYG{p}{]} \PYG{n}{Number} \PYG{n}{of} \PYG{n}{threads} \PYG{n}{used} \PYG{n}{to} \PYG{n}{run} \PYG{n}{simulation} \PYG{n}{on} \PYG{n}{CPU}\PYG{o}{.} \PYG{n}{When} \PYG{n}{using} \PYG{n}{GPU}\PYG{p}{,}
                            \PYG{n}{performance} \PYG{o+ow}{is} \PYG{n}{boosted} \PYG{n}{a} \PYG{n}{lot} \PYG{n}{so} \PYG{n}{ppn} \PYG{o+ow}{in} \PYG{n}{that} \PYG{n}{case} \PYG{o+ow}{is} \PYG{n+nb}{set} \PYG{n}{to} \PYG{l+m+mf}{1.}
\end{sphinxVerbatim}


\section{Restart simulation}
\label{\detokenize{usage/simulation_control:restart-simulation}}
\begin{sphinxVerbatim}[commandchars=\\\{\}]
\PYG{n}{restart}         \PYG{p}{(}\PYG{n+nb}{bool}\PYG{p}{)}
                \PYG{n}{yes} \PYG{p}{:} \PYG{n}{restart} \PYG{n}{simulation} \PYG{k+kn}{from} \PYG{n+nn}{checkpoint} \PYG{n}{file}\PYG{o}{.} \PYG{n}{This} \PYG{n}{can} \PYG{n}{be} \PYG{k+kc}{True}\PYG{p}{,} \PYG{l+m+mi}{1} \PYG{o+ow}{or} \PYG{n}{whatever} \PYG{n}{are} \PYG{o+ow}{not} \PYG{p}{(}\PYG{n}{FALSE}\PYG{p}{)}
                        \PYG{o+ow}{in} \PYG{n}{python} \PYG{n}{condition}\PYG{o}{.} \PYG{n}{If} \PYG{n}{this} \PYG{n}{option} \PYG{o+ow}{is} \PYG{n}{selected}\PYG{p}{,} \PYG{n}{minimize} \PYG{n}{will} \PYG{n}{be} \PYG{n}{force} \PYG{n}{to} \PYG{k+kc}{False}\PYG{o}{.}

                \PYG{n}{no} \PYG{p}{(}\PYG{n}{default}\PYG{p}{)} \PYG{p}{:} \PYG{n}{Run} \PYG{n}{simulation} \PYG{k+kn}{from} \PYG{n+nn}{beginning}\PYG{p}{,} \PYG{k}{if} \PYG{n}{this} \PYG{n}{option} \PYG{o+ow}{is} \PYG{n}{selected}\PYG{p}{,} \PYG{n}{you} \PYG{n}{can} \PYG{n}{choose} \PYG{k}{if} \PYG{n}{you} \PYG{n}{want} \PYG{n}{to} \PYG{n}{minimize} \PYG{n}{your}
                    \PYG{n}{system} \PYG{n}{before} \PYG{n}{running} \PYG{n}{simulation}\PYG{o}{.}
\PYG{o}{\PYGZhy{}}\PYG{o}{\PYGZhy{}}\PYG{o}{\PYGZhy{}}\PYG{o}{\PYGZhy{}}\PYG{o}{\PYGZhy{}}\PYG{o}{\PYGZhy{}}\PYG{o}{\PYGZhy{}}\PYG{o}{\PYGZhy{}}\PYG{o}{\PYGZhy{}}\PYG{o}{\PYGZhy{}}\PYG{o}{\PYGZhy{}}\PYG{o}{\PYGZhy{}}\PYG{o}{\PYGZhy{}}\PYG{o}{\PYGZhy{}}\PYG{o}{\PYGZhy{}}\PYG{o}{\PYGZhy{}}\PYG{o}{\PYGZhy{}}\PYG{o}{\PYGZhy{}}\PYG{o}{\PYGZhy{}}\PYG{o}{\PYGZhy{}}\PYG{o}{\PYGZhy{}}\PYG{o}{\PYGZhy{}}\PYG{o}{\PYGZhy{}}\PYG{o}{\PYGZhy{}}\PYG{o}{\PYGZhy{}}\PYG{o}{\PYGZhy{}}\PYG{o}{\PYGZhy{}}\PYG{o}{\PYGZhy{}}\PYG{o}{\PYGZhy{}}\PYG{o}{\PYGZhy{}}\PYG{o}{\PYGZhy{}}\PYG{o}{\PYGZhy{}}\PYG{o}{\PYGZhy{}}\PYG{o}{\PYGZhy{}}\PYG{o}{\PYGZhy{}}\PYG{o}{\PYGZhy{}}\PYG{o}{\PYGZhy{}}\PYG{o}{\PYGZhy{}}\PYG{o}{\PYGZhy{}}\PYG{o}{\PYGZhy{}}\PYG{o}{\PYGZhy{}}\PYG{o}{\PYGZhy{}}\PYG{o}{\PYGZhy{}}\PYG{o}{\PYGZhy{}}\PYG{o}{\PYGZhy{}}\PYG{o}{\PYGZhy{}}\PYG{o}{\PYGZhy{}}\PYG{o}{\PYGZhy{}}\PYG{o}{\PYGZhy{}}\PYG{o}{\PYGZhy{}}\PYG{o}{\PYGZhy{}}\PYG{o}{\PYGZhy{}}\PYG{o}{\PYGZhy{}}\PYG{o}{\PYGZhy{}}\PYG{o}{\PYGZhy{}}\PYG{o}{\PYGZhy{}}\PYG{o}{\PYGZhy{}}\PYG{o}{\PYGZhy{}}\PYG{o}{\PYGZhy{}}\PYG{o}{\PYGZhy{}}\PYG{o}{\PYGZhy{}}\PYG{o}{\PYGZhy{}}\PYG{o}{\PYGZhy{}}\PYG{o}{\PYGZhy{}}\PYG{o}{\PYGZhy{}}\PYG{o}{\PYGZhy{}}\PYG{o}{\PYGZhy{}}\PYG{o}{\PYGZhy{}}\PYG{o}{\PYGZhy{}}\PYG{o}{\PYGZhy{}}\PYG{o}{\PYGZhy{}}\PYG{o}{\PYGZhy{}}\PYG{o}{\PYGZhy{}}\PYG{o}{\PYGZhy{}}\PYG{o}{\PYGZhy{}}\PYG{o}{\PYGZhy{}}\PYG{o}{\PYGZhy{}}\PYG{o}{\PYGZhy{}}\PYG{o}{\PYGZhy{}}\PYG{o}{\PYGZhy{}}\PYG{o}{\PYGZhy{}}\PYG{o}{\PYGZhy{}}\PYG{o}{\PYGZhy{}}\PYG{o}{\PYGZhy{}}

\PYG{n}{minimize}        \PYG{p}{(}\PYG{n+nb}{bool}\PYG{p}{)}
                \PYG{n}{yes} \PYG{p}{(}\PYG{n}{default}\PYG{p}{)} \PYG{p}{:} \PYG{n}{perform} \PYG{n}{energy} \PYG{n}{minimization} \PYG{n}{before} \PYG{n}{run} \PYG{n}{molecular} \PYG{n}{dynamics}\PYG{o}{.}

                \PYG{n}{no} \PYG{p}{:} \PYG{n}{Not} \PYG{n}{running} \PYG{n}{energy} \PYG{n}{minimization}\PYG{o}{.} \PYG{n}{This} \PYG{o+ow}{is} \PYG{n}{default} \PYG{n}{option} \PYG{n}{when} \PYG{n}{restart} \PYG{n}{option} \PYG{o+ow}{is} \PYG{n+nb}{set} \PYG{n}{to} \PYG{n}{yes}\PYG{o}{.}
\end{sphinxVerbatim}

\sphinxstepscope


\chapter{Parameters}
\label{\detokenize{modules/parameters:module-hps.parameters.model_parameters}}\label{\detokenize{modules/parameters:parameters}}\label{\detokenize{modules/parameters::doc}}\index{module@\spxentry{module}!hps.parameters.model\_parameters@\spxentry{hps.parameters.model\_parameters}}\index{hps.parameters.model\_parameters@\spxentry{hps.parameters.model\_parameters}!module@\spxentry{module}}
\sphinxAtStartPar
Dictionary contains parameters for hps model.
First level is the model name
\begin{itemize}
\item {} 
\sphinxAtStartPar
HPS\sphinxhyphen{}Kr scale was taken from:

\end{itemize}

\sphinxAtStartPar
Dignon, G. L., Zheng, W., Kim, Y. C., Best, R. B., ; Mittal, J. (2018).
Sequence determinants of protein phase behavior from a coarse\sphinxhyphen{}grained model.
PLoS Computational Biology, 1\textendash{}23.
\sphinxurl{https://doi.org/10.1101/238170}
\begin{itemize}
\item {} 
\sphinxAtStartPar
Parameter for Nucleic acids (KR scale):

\end{itemize}

\sphinxAtStartPar
Regy, R. M., Dignon, G. L., Zheng, W., Kim, Y. C., Mittal, J. (2020).
Sequence dependent phase separation of protein\sphinxhyphen{}polynucleotide mixtures elucidated using molecular simulations.
Nucleic Acids Research, 48(22), 12593\textendash{}12603.
\sphinxurl{https://doi.org/10.1093/nar/gkaa1099}
\begin{itemize}
\item {} 
\sphinxAtStartPar
Phosphorylation version of some residues for KR scale are taken from:

\end{itemize}

\sphinxAtStartPar
Perdikari, T. M., Jovic, N., Dignon, G. L., Kim, Y. C., Fawzi, N. L.,  Mittal, J. (2021).
A predictive coarse\sphinxhyphen{}grained model for position\sphinxhyphen{}specific effects of post\sphinxhyphen{}translational modifications.
Biophysical Journal, 120(7), 1187\textendash{}1197.
\sphinxurl{https://doi.org/10.1016/j.bpj.2021.01.034}


\section{Note on hps (lambda) in urry scale:}
\label{\detokenize{modules/parameters:note-on-hps-lambda-in-urry-scale}}
\sphinxAtStartPar
\# These parameters were shifted by 0.08 from original parameters directly.

\sphinxAtStartPar
in the original paper:
Regy, R. M., Thompson, J., Kim, Y. C., ; Mittal, J. (2021).
Improved coarse\sphinxhyphen{}grained model for studying sequence dependent phase separation of disordered proteins.
Protein Science, 30(7), 1371\textendash{}1379. \sphinxurl{https://doi.org/10.1002/pro.4094}
\begin{description}
\sphinxlineitem{..math::}
\sphinxAtStartPar
lambda\_\{ij\} = muy lambda\_0\_ij \sphinxhyphen{} delta

\sphinxAtStartPar
muy=1, delta= 0.08 is the optimal set for the set of 42 proteins they studied.
lambda\_\{ij\} = lambda0\_\{ij\} \sphinxhyphen{} 0.08 = 0.5*(lambda\_i+lambda\_j) \sphinxhyphen{} 0.08 = 0.5(lambda\_i \sphinxhyphen{}0.08 + lambda\_j\sphinxhyphen{}0.08)

\end{description}

\sphinxAtStartPar
In both version, KR and Urry, we can tune directly lambda parameter in Urry by 0.08 so we can use only one equation for
two model (choose parameter when passing hps\_scale parameter)
\index{parameters (in module hps.parameters.model\_parameters)@\spxentry{parameters}\spxextra{in module hps.parameters.model\_parameters}}

\begin{fulllineitems}
\phantomsection\label{\detokenize{modules/parameters:hps.parameters.model_parameters.parameters}}
\pysigstartsignatures
\pysigline{\sphinxcode{\sphinxupquote{hps.parameters.model\_parameters.}}\sphinxbfcode{\sphinxupquote{parameters}}}
\pysigstopsignatures
\sphinxAtStartPar
dictionary contains model parameters.

\end{fulllineitems}


\sphinxstepscope


\chapter{System}
\label{\detokenize{modules/system:system}}\label{\detokenize{modules/system::doc}}\begin{description}
\sphinxlineitem{A class containing methods and parameters for generating CG systems to be simulated using the OpenMM interface.}
\sphinxAtStartPar
It offers flexibility to create default and custom CG systems and to easily modify their parameters.

\end{description}
\index{system (class in hps.core)@\spxentry{system}\spxextra{class in hps.core}}

\begin{fulllineitems}
\phantomsection\label{\detokenize{modules/system:hps.core.system}}
\pysigstartsignatures
\pysiglinewithargsret{\sphinxbfcode{\sphinxupquote{class\DUrole{w}{  }}}\sphinxcode{\sphinxupquote{hps.core.}}\sphinxbfcode{\sphinxupquote{system}}}{\emph{\DUrole{n}{structure\_path}}, \emph{\DUrole{n}{hps\_scale}}}{}
\pysigstopsignatures
\sphinxAtStartPar
A class containing methods and parameters for generating CG systems to be simulated using the OpenMM interface.
It offers flexibility to create default and custom CG systems and to easily modify their parameters.
\begin{quote}\begin{description}
\sphinxlineitem{Parameters}\begin{itemize}
\item {} 
\sphinxAtStartPar
\sphinxstyleliteralstrong{\sphinxupquote{structure\_path}} (\sphinxstyleliteralemphasis{\sphinxupquote{string}}\sphinxstyleliteralemphasis{\sphinxupquote{ {[}}}\sphinxstyleliteralemphasis{\sphinxupquote{requires}}\sphinxstyleliteralemphasis{\sphinxupquote{{]}}}) \textendash{} Name of the input PDB or CIF file

\item {} 
\sphinxAtStartPar
\sphinxstyleliteralstrong{\sphinxupquote{hps\_scale}} (\sphinxstyleliteralemphasis{\sphinxupquote{\textquotesingle{}hps\_kr\textquotesingle{}}}\sphinxstyleliteralemphasis{\sphinxupquote{,}}\sphinxstyleliteralemphasis{\sphinxupquote{\textquotesingle{}hps\_urry\textquotesingle{}}}\sphinxstyleliteralemphasis{\sphinxupquote{ {[}}}\sphinxstyleliteralemphasis{\sphinxupquote{optional}}\sphinxstyleliteralemphasis{\sphinxupquote{, }}\sphinxstyleliteralemphasis{\sphinxupquote{default=\textquotesingle{}hps\_urry\textquotesingle{}}}\sphinxstyleliteralemphasis{\sphinxupquote{{]}}}) \textendash{} Hydropathy scale. Currently, there are two models are supported.

\end{itemize}

\end{description}\end{quote}
\index{structure (hps.core.system attribute)@\spxentry{structure}\spxextra{hps.core.system attribute}}

\begin{fulllineitems}
\phantomsection\label{\detokenize{modules/system:hps.core.system.structure}}
\pysigstartsignatures
\pysigline{\sphinxbfcode{\sphinxupquote{structure}}}
\pysigstopsignatures
\sphinxAtStartPar
Object that holds the information of OpenMM PDB or CIF parsing methods.
\begin{quote}\begin{description}
\sphinxlineitem{Type}
\sphinxAtStartPar
\sphinxcode{\sphinxupquote{openmm.app.pdbfile.PDBFile or openmm.app.pdbxfile.PDBxFile}}

\end{description}\end{quote}

\end{fulllineitems}

\index{topology (hps.core.system attribute)@\spxentry{topology}\spxextra{hps.core.system attribute}}

\begin{fulllineitems}
\phantomsection\label{\detokenize{modules/system:hps.core.system.topology}}
\pysigstartsignatures
\pysigline{\sphinxbfcode{\sphinxupquote{topology}}}
\pysigstopsignatures
\sphinxAtStartPar
OpenMM topology of the model.
\begin{quote}\begin{description}
\sphinxlineitem{Type}
\sphinxAtStartPar
\sphinxcode{\sphinxupquote{openmm.app.topology.Topology}}

\end{description}\end{quote}

\end{fulllineitems}

\index{positions (hps.core.system attribute)@\spxentry{positions}\spxextra{hps.core.system attribute}}

\begin{fulllineitems}
\phantomsection\label{\detokenize{modules/system:hps.core.system.positions}}
\pysigstartsignatures
\pysigline{\sphinxbfcode{\sphinxupquote{positions}}}
\pysigstopsignatures
\sphinxAtStartPar
Atomic positions of the model.
\begin{quote}\begin{description}
\sphinxlineitem{Type}
\sphinxAtStartPar
\sphinxcode{\sphinxupquote{unit.quantity.Quantity}}

\end{description}\end{quote}

\end{fulllineitems}

\index{particles\_mass (hps.core.system attribute)@\spxentry{particles\_mass}\spxextra{hps.core.system attribute}}

\begin{fulllineitems}
\phantomsection\label{\detokenize{modules/system:hps.core.system.particles_mass}}
\pysigstartsignatures
\pysigline{\sphinxbfcode{\sphinxupquote{particles\_mass}}}
\pysigstopsignatures
\sphinxAtStartPar
Mass of each particle. If float then uniform masses are given to all
particles. If list per\sphinxhyphen{}particle masses are assigned.
\begin{quote}\begin{description}
\sphinxlineitem{Type}
\sphinxAtStartPar
\sphinxcode{\sphinxupquote{float or list}}

\end{description}\end{quote}

\end{fulllineitems}

\index{particles\_charge (hps.core.system attribute)@\spxentry{particles\_charge}\spxextra{hps.core.system attribute}}

\begin{fulllineitems}
\phantomsection\label{\detokenize{modules/system:hps.core.system.particles_charge}}
\pysigstartsignatures
\pysigline{\sphinxbfcode{\sphinxupquote{particles\_charge}}}
\pysigstopsignatures
\sphinxAtStartPar
Charge of each particle.
\begin{quote}\begin{description}
\sphinxlineitem{Type}
\sphinxAtStartPar
\sphinxcode{\sphinxupquote{list}}

\end{description}\end{quote}

\end{fulllineitems}

\index{rf\_sigma (hps.core.system attribute)@\spxentry{rf\_sigma}\spxextra{hps.core.system attribute}}

\begin{fulllineitems}
\phantomsection\label{\detokenize{modules/system:hps.core.system.rf_sigma}}
\pysigstartsignatures
\pysigline{\sphinxbfcode{\sphinxupquote{rf\_sigma}}}
\pysigstopsignatures
\sphinxAtStartPar
Sigma parameter used in the pairwise force object.
This is vdw Radius of beads
\begin{quote}\begin{description}
\sphinxlineitem{Type}
\sphinxAtStartPar
\sphinxcode{\sphinxupquote{float}}

\end{description}\end{quote}

\end{fulllineitems}

\index{atoms (hps.core.system attribute)@\spxentry{atoms}\spxextra{hps.core.system attribute}}

\begin{fulllineitems}
\phantomsection\label{\detokenize{modules/system:hps.core.system.atoms}}
\pysigstartsignatures
\pysigline{\sphinxbfcode{\sphinxupquote{atoms}}}
\pysigstopsignatures
\sphinxAtStartPar
A list of the current atoms in the model. The items are \sphinxcode{\sphinxupquote{openmm.app.topology.atoms}}
initialised classes.
\begin{quote}\begin{description}
\sphinxlineitem{Type}
\sphinxAtStartPar
\sphinxcode{\sphinxupquote{list}}

\end{description}\end{quote}

\end{fulllineitems}

\index{n\_atoms (hps.core.system attribute)@\spxentry{n\_atoms}\spxextra{hps.core.system attribute}}

\begin{fulllineitems}
\phantomsection\label{\detokenize{modules/system:hps.core.system.n_atoms}}
\pysigstartsignatures
\pysigline{\sphinxbfcode{\sphinxupquote{n\_atoms}}}
\pysigstopsignatures
\sphinxAtStartPar
Total numer of atoms in the model.
\begin{quote}\begin{description}
\sphinxlineitem{Type}
\sphinxAtStartPar
\sphinxcode{\sphinxupquote{int}}

\end{description}\end{quote}

\end{fulllineitems}

\index{bonds (hps.core.system attribute)@\spxentry{bonds}\spxextra{hps.core.system attribute}}

\begin{fulllineitems}
\phantomsection\label{\detokenize{modules/system:hps.core.system.bonds}}
\pysigstartsignatures
\pysigline{\sphinxbfcode{\sphinxupquote{bonds}}}
\pysigstopsignatures
\sphinxAtStartPar
A dict that uses bonds (2\sphinxhyphen{}tuple of \sphinxcode{\sphinxupquote{openmm.app.topology.bonds}} objects)
present in the model as keys and their forcefield properties as values.
\begin{quote}\begin{description}
\sphinxlineitem{Type}
\sphinxAtStartPar
\sphinxcode{\sphinxupquote{collections.OrderedDict}}

\end{description}\end{quote}

\end{fulllineitems}

\index{bonds\_indexes (hps.core.system attribute)@\spxentry{bonds\_indexes}\spxextra{hps.core.system attribute}}

\begin{fulllineitems}
\phantomsection\label{\detokenize{modules/system:hps.core.system.bonds_indexes}}
\pysigstartsignatures
\pysigline{\sphinxbfcode{\sphinxupquote{bonds\_indexes}}}
\pysigstopsignatures
\sphinxAtStartPar
A list containing the zero\sphinxhyphen{}based indexes of the atoms defining the bonds in the model.
\begin{quote}\begin{description}
\sphinxlineitem{Type}
\sphinxAtStartPar
\sphinxcode{\sphinxupquote{list}}

\end{description}\end{quote}

\end{fulllineitems}

\index{n\_bonds (hps.core.system attribute)@\spxentry{n\_bonds}\spxextra{hps.core.system attribute}}

\begin{fulllineitems}
\phantomsection\label{\detokenize{modules/system:hps.core.system.n_bonds}}
\pysigstartsignatures
\pysigline{\sphinxbfcode{\sphinxupquote{n\_bonds}}}
\pysigstopsignatures
\sphinxAtStartPar
Total number of bonds in the model.
\begin{quote}\begin{description}
\sphinxlineitem{Type}
\sphinxAtStartPar
\sphinxcode{\sphinxupquote{int}}

\end{description}\end{quote}

\end{fulllineitems}

\index{bonded\_exclusions\_index (hps.core.system attribute)@\spxentry{bonded\_exclusions\_index}\spxextra{hps.core.system attribute}}

\begin{fulllineitems}
\phantomsection\label{\detokenize{modules/system:hps.core.system.bonded_exclusions_index}}
\pysigstartsignatures
\pysigline{\sphinxbfcode{\sphinxupquote{bonded\_exclusions\_index}}}
\pysigstopsignatures
\sphinxAtStartPar
Exclusion rule for nonbonded force. =1 for hps\_kr and hps\_urry, =3 for hps\_ss
\begin{quote}\begin{description}
\sphinxlineitem{Type}
\sphinxAtStartPar
\sphinxcode{\sphinxupquote{int}}

\end{description}\end{quote}

\end{fulllineitems}

\index{harmonicBondForce (hps.core.system attribute)@\spxentry{harmonicBondForce}\spxextra{hps.core.system attribute}}

\begin{fulllineitems}
\phantomsection\label{\detokenize{modules/system:hps.core.system.harmonicBondForce}}
\pysigstartsignatures
\pysigline{\sphinxbfcode{\sphinxupquote{harmonicBondForce}}}
\pysigstopsignatures
\sphinxAtStartPar
Stores the OpenMM \sphinxcode{\sphinxupquote{HarmonicBondForce}} initialised\sphinxhyphen{}class. Implements
a harmonic bond potential between pairs of particles, that depends
quadratically on their distance.
\begin{quote}\begin{description}
\sphinxlineitem{Type}
\sphinxAtStartPar
\sphinxcode{\sphinxupquote{openmm.HarmonicBondForce}}

\end{description}\end{quote}

\end{fulllineitems}

\index{n\_angles (hps.core.system attribute)@\spxentry{n\_angles}\spxextra{hps.core.system attribute}}

\begin{fulllineitems}
\phantomsection\label{\detokenize{modules/system:hps.core.system.n_angles}}
\pysigstartsignatures
\pysigline{\sphinxbfcode{\sphinxupquote{n\_angles}}}
\pysigstopsignatures
\sphinxAtStartPar
Total number of angles in the model.
\begin{quote}\begin{description}
\sphinxlineitem{Type}
\sphinxAtStartPar
\sphinxcode{\sphinxupquote{int}}

\end{description}\end{quote}

\end{fulllineitems}

\index{gaussianAngleForce (hps.core.system attribute)@\spxentry{gaussianAngleForce}\spxextra{hps.core.system attribute}}

\begin{fulllineitems}
\phantomsection\label{\detokenize{modules/system:hps.core.system.gaussianAngleForce}}
\pysigstartsignatures
\pysigline{\sphinxbfcode{\sphinxupquote{gaussianAngleForce}}}
\pysigstopsignatures
\sphinxAtStartPar
Stores the OpenMM \sphinxcode{\sphinxupquote{CustomAngleForce}} initialised\sphinxhyphen{}class. Implements
a Gaussian angle bond potential between pairs of three particles.
\begin{quote}\begin{description}
\sphinxlineitem{Type}
\sphinxAtStartPar
\sphinxcode{\sphinxupquote{openmm.CustomAngleForce}}

\end{description}\end{quote}

\end{fulllineitems}

\index{n\_torsions (hps.core.system attribute)@\spxentry{n\_torsions}\spxextra{hps.core.system attribute}}

\begin{fulllineitems}
\phantomsection\label{\detokenize{modules/system:hps.core.system.n_torsions}}
\pysigstartsignatures
\pysigline{\sphinxbfcode{\sphinxupquote{n\_torsions}}}
\pysigstopsignatures
\sphinxAtStartPar
Total number of torsion angles in the model.
\begin{quote}\begin{description}
\sphinxlineitem{Type}
\sphinxAtStartPar
\sphinxcode{\sphinxupquote{int}}

\end{description}\end{quote}

\end{fulllineitems}

\index{gaussianTorsionForce (hps.core.system attribute)@\spxentry{gaussianTorsionForce}\spxextra{hps.core.system attribute}}

\begin{fulllineitems}
\phantomsection\label{\detokenize{modules/system:hps.core.system.gaussianTorsionForce}}
\pysigstartsignatures
\pysigline{\sphinxbfcode{\sphinxupquote{gaussianTorsionForce}}}
\pysigstopsignatures
\sphinxAtStartPar
Stores the OpenMM \sphinxcode{\sphinxupquote{CustomTorsionForce}} initialised\sphinxhyphen{}class. Implements
a Gaussian torsion angle bond potential between pairs of four particles.
\begin{quote}\begin{description}
\sphinxlineitem{Type}
\sphinxAtStartPar
\sphinxcode{\sphinxupquote{openmm.CustomTorsionForce}}

\end{description}\end{quote}

\end{fulllineitems}

\index{yukawaForce (hps.core.system attribute)@\spxentry{yukawaForce}\spxextra{hps.core.system attribute}}

\begin{fulllineitems}
\phantomsection\label{\detokenize{modules/system:hps.core.system.yukawaForce}}
\pysigstartsignatures
\pysigline{\sphinxbfcode{\sphinxupquote{yukawaForce}}}
\pysigstopsignatures
\sphinxAtStartPar
Stores the OpenMM \sphinxcode{\sphinxupquote{CustomNonbondedForce}} initialized\sphinxhyphen{}class.
Implements the Debye\sphinxhyphen{}Huckle potential.
\begin{quote}\begin{description}
\sphinxlineitem{Type}
\sphinxAtStartPar
\sphinxcode{\sphinxupquote{openmm.CustomNonbondedForce}}

\end{description}\end{quote}

\end{fulllineitems}

\index{ashbaugh\_HatchForce (hps.core.system attribute)@\spxentry{ashbaugh\_HatchForce}\spxextra{hps.core.system attribute}}

\begin{fulllineitems}
\phantomsection\label{\detokenize{modules/system:hps.core.system.ashbaugh_HatchForce}}
\pysigstartsignatures
\pysigline{\sphinxbfcode{\sphinxupquote{ashbaugh\_HatchForce}}}
\pysigstopsignatures
\sphinxAtStartPar
Stores the OpenMM \sphinxcode{\sphinxupquote{CustomNonbondedForce}} initialized\sphinxhyphen{}class. Implements the pairwise short\sphinxhyphen{}range
potential.
\begin{quote}\begin{description}
\sphinxlineitem{Type}
\sphinxAtStartPar
\sphinxcode{\sphinxupquote{openmm.CustomNonbondedForce}}

\end{description}\end{quote}

\end{fulllineitems}

\index{forceGroups (hps.core.system attribute)@\spxentry{forceGroups}\spxextra{hps.core.system attribute}}

\begin{fulllineitems}
\phantomsection\label{\detokenize{modules/system:hps.core.system.forceGroups}}
\pysigstartsignatures
\pysigline{\sphinxbfcode{\sphinxupquote{forceGroups}}}
\pysigstopsignatures
\sphinxAtStartPar
A dict that uses force names as keys and their corresponding force
as values.
\begin{quote}\begin{description}
\sphinxlineitem{Type}
\sphinxAtStartPar
\sphinxcode{\sphinxupquote{collections.OrderedDict}}

\end{description}\end{quote}

\end{fulllineitems}

\index{system (hps.core.system attribute)@\spxentry{system}\spxextra{hps.core.system attribute}}

\begin{fulllineitems}
\phantomsection\label{\detokenize{modules/system:hps.core.system.system}}
\pysigstartsignatures
\pysigline{\sphinxbfcode{\sphinxupquote{system}}}
\pysigstopsignatures
\sphinxAtStartPar
Stores the OpenMM System initialised class. It stores all the forcefield
information for the hps model.
\begin{quote}\begin{description}
\sphinxlineitem{Type}
\sphinxAtStartPar
\sphinxcode{\sphinxupquote{openmm.System}}

\end{description}\end{quote}

\end{fulllineitems}

\index{loadForcefieldFromFile() (hps.core.system method)@\spxentry{loadForcefieldFromFile()}\spxextra{hps.core.system method}}

\begin{fulllineitems}
\phantomsection\label{\detokenize{modules/system:hps.core.system.loadForcefieldFromFile}}
\pysigstartsignatures
\pysiglinewithargsret{\sphinxbfcode{\sphinxupquote{loadForcefieldFromFile}}}{}{}
\pysigstopsignatures
\sphinxAtStartPar
Loads forcefield parameters from a force field file written with
the \sphinxcode{\sphinxupquote{dumpForceFieldData()}} method.

\end{fulllineitems}

\index{\_\_init\_\_() (hps.core.system method)@\spxentry{\_\_init\_\_()}\spxextra{hps.core.system method}}

\begin{fulllineitems}
\phantomsection\label{\detokenize{modules/system:hps.core.system.__init__}}
\pysigstartsignatures
\pysiglinewithargsret{\sphinxbfcode{\sphinxupquote{\_\_init\_\_}}}{\emph{\DUrole{n}{structure\_path}}, \emph{\DUrole{n}{hps\_scale}}}{}
\pysigstopsignatures
\sphinxAtStartPar
Initialises the hps OpenMM system class.
\begin{quote}\begin{description}
\sphinxlineitem{Parameters}\begin{itemize}
\item {} 
\sphinxAtStartPar
\sphinxstyleliteralstrong{\sphinxupquote{structure\_path}} (\sphinxstyleliteralemphasis{\sphinxupquote{string}}\sphinxstyleliteralemphasis{\sphinxupquote{ {[}}}\sphinxstyleliteralemphasis{\sphinxupquote{requires}}\sphinxstyleliteralemphasis{\sphinxupquote{{]}}}) \textendash{} Name of the input PDB or CIF file

\item {} 
\sphinxAtStartPar
\sphinxstyleliteralstrong{\sphinxupquote{hps\_scale}} (\sphinxstyleliteralemphasis{\sphinxupquote{\textquotesingle{}hps\_kr\textquotesingle{}}}\sphinxstyleliteralemphasis{\sphinxupquote{,}}\sphinxstyleliteralemphasis{\sphinxupquote{\textquotesingle{}hps\_urry\textquotesingle{}}}\sphinxstyleliteralemphasis{\sphinxupquote{, or }}\sphinxstyleliteralemphasis{\sphinxupquote{\textquotesingle{}hps\_ss\textquotesingle{}}}\sphinxstyleliteralemphasis{\sphinxupquote{ {[}}}\sphinxstyleliteralemphasis{\sphinxupquote{optional}}\sphinxstyleliteralemphasis{\sphinxupquote{, }}\sphinxstyleliteralemphasis{\sphinxupquote{default=\textquotesingle{}hps\_urry\textquotesingle{}}}\sphinxstyleliteralemphasis{\sphinxupquote{{]}}}) \textendash{} Hydropathy scale. Currently, there are three models are supported.

\end{itemize}

\sphinxlineitem{Return type}
\sphinxAtStartPar
None

\end{description}\end{quote}

\end{fulllineitems}

\index{getAtoms() (hps.core.system method)@\spxentry{getAtoms()}\spxextra{hps.core.system method}}

\begin{fulllineitems}
\phantomsection\label{\detokenize{modules/system:hps.core.system.getAtoms}}
\pysigstartsignatures
\pysiglinewithargsret{\sphinxbfcode{\sphinxupquote{getAtoms}}}{}{}
\pysigstopsignatures
\sphinxAtStartPar
Reads atoms from topology, adds them to the main class and sorts them
into a dictionary to store their forcefield properties.

\sphinxAtStartPar
After getCAlphaOnly, C\sphinxhyphen{}alpha atoms are stored on \sphinxcode{\sphinxupquote{self.topology only}}.
We need to add them to atoms attribute and system also.
Adds \sphinxcode{\sphinxupquote{atoms}} in the \sphinxcode{\sphinxupquote{OpenMM topology}} instance to the \sphinxcode{\sphinxupquote{hpsOpenMM system}} class.
\begin{quote}\begin{description}
\sphinxlineitem{Return type}
\sphinxAtStartPar
None

\end{description}\end{quote}

\end{fulllineitems}

\index{getBonds() (hps.core.system method)@\spxentry{getBonds()}\spxextra{hps.core.system method}}

\begin{fulllineitems}
\phantomsection\label{\detokenize{modules/system:hps.core.system.getBonds}}
\pysigstartsignatures
\pysiglinewithargsret{\sphinxbfcode{\sphinxupquote{getBonds}}}{\emph{\DUrole{n}{except\_chains}\DUrole{o}{=}\DUrole{default_value}{None}}}{}
\pysigstopsignatures
\sphinxAtStartPar
Reads bonds from topology, adds them to the main class and sorts them
into a dictionary to store their forcefield properties.

\sphinxAtStartPar
Adds \sphinxcode{\sphinxupquote{bonds}} in the \sphinxcode{\sphinxupquote{OpenMM topology}} instance to the \sphinxcode{\sphinxupquote{hpsOpenMM system}} class.
\begin{quote}\begin{description}
\sphinxlineitem{Parameters}
\sphinxAtStartPar
\sphinxstyleliteralstrong{\sphinxupquote{except\_chains}} (\sphinxstyleliteralemphasis{\sphinxupquote{String}}\sphinxstyleliteralemphasis{\sphinxupquote{ {[}}}\sphinxstyleliteralemphasis{\sphinxupquote{optional}}\sphinxstyleliteralemphasis{\sphinxupquote{{]}}}) \textendash{} 

\sphinxlineitem{Return type}
\sphinxAtStartPar
None

\end{description}\end{quote}

\end{fulllineitems}

\index{setBondForceConstants() (hps.core.system method)@\spxentry{setBondForceConstants()}\spxextra{hps.core.system method}}

\begin{fulllineitems}
\phantomsection\label{\detokenize{modules/system:hps.core.system.setBondForceConstants}}
\pysigstartsignatures
\pysiglinewithargsret{\sphinxbfcode{\sphinxupquote{setBondForceConstants}}}{\emph{\DUrole{n}{bond\_force\_constant}}}{}
\pysigstopsignatures
\sphinxAtStartPar
Change the forcefield parameters for bonded terms.

\sphinxAtStartPar
Set the harmonic bond constant force parameters. The input can be
a float, to set the same parameter for all force interactions, or
a list, to define a unique parameter for each force interaction.
\begin{quote}\begin{description}
\sphinxlineitem{Parameters}
\sphinxAtStartPar
\sphinxstyleliteralstrong{\sphinxupquote{bond\_force\_constant}} (\sphinxstyleliteralemphasis{\sphinxupquote{float}}\sphinxstyleliteralemphasis{\sphinxupquote{ or }}\sphinxstyleliteralemphasis{\sphinxupquote{list}}) \textendash{} Parameter(s) to set up for the harmonic bond forces.

\sphinxlineitem{Return type}
\sphinxAtStartPar
None

\end{description}\end{quote}

\end{fulllineitems}

\index{setParticlesRadii() (hps.core.system method)@\spxentry{setParticlesRadii()}\spxextra{hps.core.system method}}

\begin{fulllineitems}
\phantomsection\label{\detokenize{modules/system:hps.core.system.setParticlesRadii}}
\pysigstartsignatures
\pysiglinewithargsret{\sphinxbfcode{\sphinxupquote{setParticlesRadii}}}{\emph{\DUrole{n}{particles\_radii}}}{}
\pysigstopsignatures
\sphinxAtStartPar
Change the excluded volume radius parameter for each atom in the system.

\sphinxAtStartPar
Set the radii of the particles in the system. The input can be a
float, to set the same radius for all particles, or a list, to define
a unique radius for each particle.
\begin{quote}\begin{description}
\sphinxlineitem{Parameters}
\sphinxAtStartPar
\sphinxstyleliteralstrong{\sphinxupquote{particles\_radii}} (\sphinxstyleliteralemphasis{\sphinxupquote{float}}\sphinxstyleliteralemphasis{\sphinxupquote{ or }}\sphinxstyleliteralemphasis{\sphinxupquote{list}}) \textendash{} Radii values to add for the particles in the hpsOpenMM system class.

\sphinxlineitem{Return type}
\sphinxAtStartPar
None

\end{description}\end{quote}

\end{fulllineitems}

\index{setParticlesCharge() (hps.core.system method)@\spxentry{setParticlesCharge()}\spxextra{hps.core.system method}}

\begin{fulllineitems}
\phantomsection\label{\detokenize{modules/system:hps.core.system.setParticlesCharge}}
\pysigstartsignatures
\pysiglinewithargsret{\sphinxbfcode{\sphinxupquote{setParticlesCharge}}}{\emph{\DUrole{n}{particles\_charge}}}{}
\pysigstopsignatures
\sphinxAtStartPar
Set the charge of the particles in the system. The input can be a
float, to set the same charge for all particles, or a list, to define
a unique charge for each particle.
\begin{quote}\begin{description}
\sphinxlineitem{Parameters}
\sphinxAtStartPar
\sphinxstyleliteralstrong{\sphinxupquote{particles\_charge}} (\sphinxstyleliteralemphasis{\sphinxupquote{float}}\sphinxstyleliteralemphasis{\sphinxupquote{ or }}\sphinxstyleliteralemphasis{\sphinxupquote{list}}) \textendash{} Charge values to add for the particles in the hpsOpenMM system class.

\sphinxlineitem{Return type}
\sphinxAtStartPar
None

\end{description}\end{quote}

\end{fulllineitems}

\index{setParticlesHPS() (hps.core.system method)@\spxentry{setParticlesHPS()}\spxextra{hps.core.system method}}

\begin{fulllineitems}
\phantomsection\label{\detokenize{modules/system:hps.core.system.setParticlesHPS}}
\pysigstartsignatures
\pysiglinewithargsret{\sphinxbfcode{\sphinxupquote{setParticlesHPS}}}{\emph{\DUrole{n}{particles\_hps}}}{}
\pysigstopsignatures
\sphinxAtStartPar
Set the hydropathy scale of the particles in the system. The input can be a
float, to set the same hydropathy for all particles, or a list, to define
a unique hydropathy for each particle.
\begin{quote}\begin{description}
\sphinxlineitem{Parameters}
\sphinxAtStartPar
\sphinxstyleliteralstrong{\sphinxupquote{particles\_hps}} (\sphinxstyleliteralemphasis{\sphinxupquote{float}}\sphinxstyleliteralemphasis{\sphinxupquote{ or }}\sphinxstyleliteralemphasis{\sphinxupquote{list}}) \textendash{} HPS scale values to add for the particles in the hpsOpenMM system class.

\sphinxlineitem{Return type}
\sphinxAtStartPar
None

\end{description}\end{quote}

\end{fulllineitems}

\index{addYukawaForces() (hps.core.system method)@\spxentry{addYukawaForces()}\spxextra{hps.core.system method}}

\begin{fulllineitems}
\phantomsection\label{\detokenize{modules/system:hps.core.system.addYukawaForces}}
\pysigstartsignatures
\pysiglinewithargsret{\sphinxbfcode{\sphinxupquote{addYukawaForces}}}{\emph{\DUrole{n}{use\_pbc}\DUrole{p}{:}\DUrole{w}{  }\DUrole{n}{bool}}}{{ $\rightarrow$ None}}
\pysigstopsignatures
\sphinxAtStartPar
Creates a nonbonded force term for electrostatic interaction DH potential.

\sphinxAtStartPar
Creates an \sphinxcode{\sphinxupquote{openmm.CustomNonbondedForce()}} object with the parameters
sigma and epsilon given to this method. The custom non\sphinxhyphen{}bonded force
is initialized with the formula:
\begin{equation*}
\begin{split}energy = f \times \frac{q_1q_2}{\epsilon_r \times r}\times e^{(-r/lD)}\end{split}
\end{equation*}
\sphinxAtStartPar
where \(f=\frac{1}{4\pi\epsilon_0}=138.935458\) is the factor for short to convert dimensionless
in calculation to \(kj.nm/(mol\times e^2)\) unit.

\sphinxAtStartPar
\(\epsilon_r=80\): Dielectric constant of water at 100mM mono\sphinxhyphen{}valent ion

\sphinxAtStartPar
The force object is stored at the \sphinxcode{\sphinxupquote{yukawaForce}} attribute.
\begin{quote}\begin{description}
\sphinxlineitem{Parameters}
\sphinxAtStartPar
\sphinxstyleliteralstrong{\sphinxupquote{use\_pbc}} (\sphinxstyleliteralemphasis{\sphinxupquote{(}}\sphinxstyleliteralemphasis{\sphinxupquote{bool}}\sphinxstyleliteralemphasis{\sphinxupquote{) }}\sphinxstyleliteralemphasis{\sphinxupquote{whether use PBC}}\sphinxstyleliteralemphasis{\sphinxupquote{, }}\sphinxstyleliteralemphasis{\sphinxupquote{cutoff periodic boundary condition}}) \textendash{} 

\sphinxlineitem{Return type}
\sphinxAtStartPar
None

\end{description}\end{quote}

\end{fulllineitems}

\index{addAshbaughHatchForces() (hps.core.system method)@\spxentry{addAshbaughHatchForces()}\spxextra{hps.core.system method}}

\begin{fulllineitems}
\phantomsection\label{\detokenize{modules/system:hps.core.system.addAshbaughHatchForces}}
\pysigstartsignatures
\pysiglinewithargsret{\sphinxbfcode{\sphinxupquote{addAshbaughHatchForces}}}{\emph{\DUrole{n}{use\_pbc}\DUrole{p}{:}\DUrole{w}{  }\DUrole{n}{bool}}}{{ $\rightarrow$ None}}
\pysigstopsignatures
\sphinxAtStartPar
Creates a nonbonded force term for pairwise interaction (customize LJ 12\sphinxhyphen{}6 potential).

\sphinxAtStartPar
Creates an \sphinxcode{\sphinxupquote{openmm.CustomNonbondedForce()}} object with the parameters
sigma and epsilon given to this method. The custom non\sphinxhyphen{}bonded force
is initialized with the formula: (note: hps here is \(\lambda_{ij}^{0}\) in the paper)

\sphinxAtStartPar
Unlike \sphinxcode{\sphinxupquote{BondForce}} class, where we specify index for atoms pair to add bond, it means
that number of bondForces may differ from number of particle.
\sphinxcode{\sphinxupquote{NonBondedForce}} is added to all particles, hence we don’t need to pass the \sphinxcode{\sphinxupquote{atom index}}.
\begin{align*}\!\begin{aligned}
\Phi_{i,j}^{vdw}(r) = step(2^{1/6}\sigma_{ij}-r) \times
\left( 4\epsilon\left[\left(\frac{\sigma_{ij}}{r}\right)^{12}-
\left(\frac{\sigma_{ij}}{r}\right)^{6}\right]+(1-\lambda_{ij})\epsilon\right)\\
+ \left[1-step(2^{1/6}\sigma_{ij}-r)\right]\times\left[(\lambda_{ij})\times 4\epsilon
\left[\left(\frac{\sigma_{ij}}{r}\right)^{12}-\left(\frac{\sigma_{ij}}{r}\right)^6\right]\right]\\
\end{aligned}\end{align*}
\sphinxAtStartPar
Here, \(\sigma= \frac{(\sigma_1+\sigma_2)}{2}; \lambda_{ij}^{0}=\frac{(\lambda_i+\lambda_j)}{2};
\epsilon = 0.8368 kj/mol\)

\sphinxAtStartPar
The force object is stored at the \sphinxcode{\sphinxupquote{ashbaugh\_HatchForce}} attribute.
\begin{description}
\sphinxlineitem{epsilon}{[}float{]}
\sphinxAtStartPar
Value of the epsilon constant in the energy function.

\sphinxlineitem{sigma}{[}float or list{]}
\sphinxAtStartPar
Value of the sigma constant (in nm) in the energy function. If float the
same sigma value is used for every particle. If list a unique
parameter is given for each particle.

\sphinxlineitem{cutoff}{[}float{]}
\sphinxAtStartPar
The cutoff distance (in nm) being used for the non\sphinxhyphen{}bonded interactions.

\end{description}
\begin{quote}\begin{description}
\sphinxlineitem{Parameters}
\sphinxAtStartPar
\sphinxstyleliteralstrong{\sphinxupquote{use\_pbc}} (\sphinxstyleliteralemphasis{\sphinxupquote{bool. Whether use PBC}}\sphinxstyleliteralemphasis{\sphinxupquote{, }}\sphinxstyleliteralemphasis{\sphinxupquote{cutoff periodic boundary condition}}) \textendash{} 

\sphinxlineitem{Return type}
\sphinxAtStartPar
None

\end{description}\end{quote}

\end{fulllineitems}

\index{createSystemObject() (hps.core.system method)@\spxentry{createSystemObject()}\spxextra{hps.core.system method}}

\begin{fulllineitems}
\phantomsection\label{\detokenize{modules/system:hps.core.system.createSystemObject}}
\pysigstartsignatures
\pysiglinewithargsret{\sphinxbfcode{\sphinxupquote{createSystemObject}}}{\emph{\DUrole{n}{check\_bond\_distances}\DUrole{p}{:}\DUrole{w}{  }\DUrole{n}{bool}\DUrole{w}{  }\DUrole{o}{=}\DUrole{w}{  }\DUrole{default_value}{True}}, \emph{\DUrole{n}{minimize}\DUrole{p}{:}\DUrole{w}{  }\DUrole{n}{bool}\DUrole{w}{  }\DUrole{o}{=}\DUrole{w}{  }\DUrole{default_value}{False}}, \emph{\DUrole{n}{check\_large\_forces}\DUrole{p}{:}\DUrole{w}{  }\DUrole{n}{bool}\DUrole{w}{  }\DUrole{o}{=}\DUrole{w}{  }\DUrole{default_value}{True}}, \emph{\DUrole{n}{force\_threshold}\DUrole{p}{:}\DUrole{w}{  }\DUrole{n}{float}\DUrole{w}{  }\DUrole{o}{=}\DUrole{w}{  }\DUrole{default_value}{10.0}}, \emph{\DUrole{n}{bond\_threshold}\DUrole{p}{:}\DUrole{w}{  }\DUrole{n}{float}\DUrole{w}{  }\DUrole{o}{=}\DUrole{w}{  }\DUrole{default_value}{0.5}}}{{ $\rightarrow$ None}}
\pysigstopsignatures
\sphinxAtStartPar
Creates OpenMM system object adding particles, masses and forces.
It also groups the added forces into Force\sphinxhyphen{}Groups for the hpsReporter
class.

\sphinxAtStartPar
Creates an \sphinxcode{\sphinxupquote{openmm.System()}} object using the force field parameters
given to the ‘system’ class. It adds particles, forces and
creates a force group for each force object. Optionally the method
can check for large bond distances (default) and minimize the atomic
positions if large forces are found in any atom (default False).
\begin{quote}\begin{description}
\sphinxlineitem{Parameters}\begin{itemize}
\item {} 
\sphinxAtStartPar
\sphinxstyleliteralstrong{\sphinxupquote{minimize}} (\sphinxstyleliteralemphasis{\sphinxupquote{boolean}}\sphinxstyleliteralemphasis{\sphinxupquote{ (}}\sphinxstyleliteralemphasis{\sphinxupquote{False}}\sphinxstyleliteralemphasis{\sphinxupquote{)}}) \textendash{} Whether to minimize the system if large forces are found.

\item {} 
\sphinxAtStartPar
\sphinxstyleliteralstrong{\sphinxupquote{check\_bond\_distances}} (\sphinxstyleliteralemphasis{\sphinxupquote{boolean}}\sphinxstyleliteralemphasis{\sphinxupquote{ (}}\sphinxstyleliteralemphasis{\sphinxupquote{True}}\sphinxstyleliteralemphasis{\sphinxupquote{)}}) \textendash{} Whether to check for large bond distances.

\item {} 
\sphinxAtStartPar
\sphinxstyleliteralstrong{\sphinxupquote{check\_large\_forces}} (\sphinxstyleliteralemphasis{\sphinxupquote{boolean}}\sphinxstyleliteralemphasis{\sphinxupquote{ (}}\sphinxstyleliteralemphasis{\sphinxupquote{False}}\sphinxstyleliteralemphasis{\sphinxupquote{)}}) \textendash{} Whether to print force summary of force groups

\item {} 
\sphinxAtStartPar
\sphinxstyleliteralstrong{\sphinxupquote{force\_threshold}} (\sphinxstyleliteralemphasis{\sphinxupquote{float}}\sphinxstyleliteralemphasis{\sphinxupquote{ (}}\sphinxstyleliteralemphasis{\sphinxupquote{10.0}}\sphinxstyleliteralemphasis{\sphinxupquote{)}}) \textendash{} Threshold to check for large forces.

\item {} 
\sphinxAtStartPar
\sphinxstyleliteralstrong{\sphinxupquote{bond\_threshold}} (\sphinxstyleliteralemphasis{\sphinxupquote{float}}\sphinxstyleliteralemphasis{\sphinxupquote{ (}}\sphinxstyleliteralemphasis{\sphinxupquote{0.5}}\sphinxstyleliteralemphasis{\sphinxupquote{)}}) \textendash{} Threshold to check for large bond distances.

\end{itemize}

\sphinxlineitem{Return type}
\sphinxAtStartPar
None

\end{description}\end{quote}

\end{fulllineitems}

\index{checkBondDistances() (hps.core.system method)@\spxentry{checkBondDistances()}\spxextra{hps.core.system method}}

\begin{fulllineitems}
\phantomsection\label{\detokenize{modules/system:hps.core.system.checkBondDistances}}
\pysigstartsignatures
\pysiglinewithargsret{\sphinxbfcode{\sphinxupquote{checkBondDistances}}}{\emph{\DUrole{n}{threshold}\DUrole{p}{:}\DUrole{w}{  }\DUrole{n}{float}\DUrole{w}{  }\DUrole{o}{=}\DUrole{w}{  }\DUrole{default_value}{0.5}}}{{ $\rightarrow$ None}}
\pysigstopsignatures
\sphinxAtStartPar
Searches for large bond distances for the atom pairs defined in
the ‘bonds’ attribute. It raises an error when large bonds are found.
\begin{quote}\begin{description}
\sphinxlineitem{Parameters}
\sphinxAtStartPar
\sphinxstyleliteralstrong{\sphinxupquote{threshold}} (\sphinxstyleliteralemphasis{\sphinxupquote{(}}\sphinxstyleliteralemphasis{\sphinxupquote{float}}\sphinxstyleliteralemphasis{\sphinxupquote{, }}\sphinxstyleliteralemphasis{\sphinxupquote{default=0.5 nm}}\sphinxstyleliteralemphasis{\sphinxupquote{)}}) \textendash{} Threshold to check for large bond distances.

\sphinxlineitem{Return type}
\sphinxAtStartPar
None

\end{description}\end{quote}

\end{fulllineitems}

\index{checkLargeForces() (hps.core.system method)@\spxentry{checkLargeForces()}\spxextra{hps.core.system method}}

\begin{fulllineitems}
\phantomsection\label{\detokenize{modules/system:hps.core.system.checkLargeForces}}
\pysigstartsignatures
\pysiglinewithargsret{\sphinxbfcode{\sphinxupquote{checkLargeForces}}}{\emph{\DUrole{n}{minimize}\DUrole{p}{:}\DUrole{w}{  }\DUrole{n}{bool}\DUrole{w}{  }\DUrole{o}{=}\DUrole{w}{  }\DUrole{default_value}{False}}, \emph{\DUrole{n}{threshold}\DUrole{p}{:}\DUrole{w}{  }\DUrole{n}{float}\DUrole{w}{  }\DUrole{o}{=}\DUrole{w}{  }\DUrole{default_value}{10}}}{{ $\rightarrow$ None}}
\pysigstopsignatures
\sphinxAtStartPar
Prints the hps system energies of the input configuration of the
system. It optionally checks for large forces acting upon all
particles in the hps system and iteratively minimizes the system
configuration until no forces larger than a threshold are found.
\begin{quote}\begin{description}
\sphinxlineitem{Parameters}\begin{itemize}
\item {} 
\sphinxAtStartPar
\sphinxstyleliteralstrong{\sphinxupquote{threshold}} (\sphinxstyleliteralemphasis{\sphinxupquote{(}}\sphinxstyleliteralemphasis{\sphinxupquote{float}}\sphinxstyleliteralemphasis{\sphinxupquote{, }}\sphinxstyleliteralemphasis{\sphinxupquote{default=10}}\sphinxstyleliteralemphasis{\sphinxupquote{)}}) \textendash{} Threshold to check for large forces.

\item {} 
\sphinxAtStartPar
\sphinxstyleliteralstrong{\sphinxupquote{minimize}} (\sphinxstyleliteralemphasis{\sphinxupquote{(}}\sphinxstyleliteralemphasis{\sphinxupquote{bool}}\sphinxstyleliteralemphasis{\sphinxupquote{, }}\sphinxstyleliteralemphasis{\sphinxupquote{default= False}}\sphinxstyleliteralemphasis{\sphinxupquote{)}}) \textendash{} Whether to iteratively minimize the system until all forces are lower or equal to
the threshold value.

\end{itemize}

\sphinxlineitem{Return type}
\sphinxAtStartPar
None

\end{description}\end{quote}

\end{fulllineitems}

\index{addParticles() (hps.core.system method)@\spxentry{addParticles()}\spxextra{hps.core.system method}}

\begin{fulllineitems}
\phantomsection\label{\detokenize{modules/system:hps.core.system.addParticles}}
\pysigstartsignatures
\pysiglinewithargsret{\sphinxbfcode{\sphinxupquote{addParticles}}}{}{{ $\rightarrow$ None}}
\pysigstopsignatures
\sphinxAtStartPar
Add particles to the system OpenMM class instance.

\sphinxAtStartPar
Add a particle to the system for each atom in it. The mass
of each particle is set up with the values in the \sphinxcode{\sphinxupquote{particles\_mass}}
attribute.

\end{fulllineitems}

\index{addSystemForces() (hps.core.system method)@\spxentry{addSystemForces()}\spxextra{hps.core.system method}}

\begin{fulllineitems}
\phantomsection\label{\detokenize{modules/system:hps.core.system.addSystemForces}}
\pysigstartsignatures
\pysiglinewithargsret{\sphinxbfcode{\sphinxupquote{addSystemForces}}}{}{{ $\rightarrow$ None}}
\pysigstopsignatures
\sphinxAtStartPar
Add forces to the system OpenMM class instance. It also save
names for the added forces to include them in the reporter class.

\sphinxAtStartPar
Adds generated forces to the system, also adding
a force group to the \sphinxcode{\sphinxupquote{forceGroups}} attribute dictionary.

\end{fulllineitems}

\index{dumpStructure() (hps.core.system method)@\spxentry{dumpStructure()}\spxextra{hps.core.system method}}

\begin{fulllineitems}
\phantomsection\label{\detokenize{modules/system:hps.core.system.dumpStructure}}
\pysigstartsignatures
\pysiglinewithargsret{\sphinxbfcode{\sphinxupquote{dumpStructure}}}{\emph{\DUrole{n}{output\_file}\DUrole{p}{:}\DUrole{w}{  }\DUrole{n}{str}}}{{ $\rightarrow$ None}}
\pysigstopsignatures
\sphinxAtStartPar
Writes a structure file of the system in its current state.

\sphinxAtStartPar
Writes a PDB file containing the currently defined CG system atoms and its positions.
\begin{quote}\begin{description}
\sphinxlineitem{Parameters}
\sphinxAtStartPar
\sphinxstyleliteralstrong{\sphinxupquote{output\_file}} (\sphinxstyleliteralemphasis{\sphinxupquote{string}}) \textendash{} name of the PDB output file.

\sphinxlineitem{Return type}
\sphinxAtStartPar
None

\end{description}\end{quote}

\end{fulllineitems}

\index{dumpTopology() (hps.core.system method)@\spxentry{dumpTopology()}\spxextra{hps.core.system method}}

\begin{fulllineitems}
\phantomsection\label{\detokenize{modules/system:hps.core.system.dumpTopology}}
\pysigstartsignatures
\pysiglinewithargsret{\sphinxbfcode{\sphinxupquote{dumpTopology}}}{\emph{\DUrole{n}{output\_file}\DUrole{p}{:}\DUrole{w}{  }\DUrole{n}{str}}}{{ $\rightarrow$ None}}
\pysigstopsignatures
\sphinxAtStartPar
Writes a topology file of the system in PSF format, this is used for visualization and post\sphinxhyphen{}analysis.

\sphinxAtStartPar
Writes a file containing the current topology in the
hpsOpenMM system. This file contains topology of system, used in visualization and analysis.

\sphinxAtStartPar
Here, we used \sphinxcode{\sphinxupquote{parmed}} to load \sphinxcode{\sphinxupquote{openMM topology}} and \sphinxcode{\sphinxupquote{openMM system}} to create
\sphinxcode{\sphinxupquote{Structure}} object in \sphinxcode{\sphinxupquote{parmed}}.
Because parmed doesn’t automatically recognize \sphinxcode{\sphinxupquote{charge}}, \sphinxcode{\sphinxupquote{mass}} of atoms by their name.
We need to set \sphinxcode{\sphinxupquote{charge}}, \sphinxcode{\sphinxupquote{mass}} back to residues properties.
\begin{quote}\begin{description}
\sphinxlineitem{Parameters}
\sphinxAtStartPar
\sphinxstyleliteralstrong{\sphinxupquote{output\_file}} (\sphinxstyleliteralemphasis{\sphinxupquote{string}}\sphinxstyleliteralemphasis{\sphinxupquote{ {[}}}\sphinxstyleliteralemphasis{\sphinxupquote{requires}}\sphinxstyleliteralemphasis{\sphinxupquote{{]}}}) \textendash{} name of the output PSF file.

\sphinxlineitem{Return type}
\sphinxAtStartPar
None

\end{description}\end{quote}

\end{fulllineitems}

\index{dumpForceFieldData() (hps.core.system method)@\spxentry{dumpForceFieldData()}\spxextra{hps.core.system method}}

\begin{fulllineitems}
\phantomsection\label{\detokenize{modules/system:hps.core.system.dumpForceFieldData}}
\pysigstartsignatures
\pysiglinewithargsret{\sphinxbfcode{\sphinxupquote{dumpForceFieldData}}}{\emph{\DUrole{n}{output\_file}\DUrole{p}{:}\DUrole{w}{  }\DUrole{n}{str}}}{{ $\rightarrow$ None}}
\pysigstopsignatures
\sphinxAtStartPar
Writes to a file the parameters of the forcefield.

\sphinxAtStartPar
Writes a file containing the current forcefield parameters in the
CG system.
\begin{quote}\begin{description}
\sphinxlineitem{Parameters}
\sphinxAtStartPar
\sphinxstyleliteralstrong{\sphinxupquote{output\_file}} (\sphinxstyleliteralemphasis{\sphinxupquote{string}}\sphinxstyleliteralemphasis{\sphinxupquote{ {[}}}\sphinxstyleliteralemphasis{\sphinxupquote{requires}}\sphinxstyleliteralemphasis{\sphinxupquote{{]}}}) \textendash{} name of the output file.

\sphinxlineitem{Return type}
\sphinxAtStartPar
None

\end{description}\end{quote}

\end{fulllineitems}

\index{setCAMassPerResidueType() (hps.core.system method)@\spxentry{setCAMassPerResidueType()}\spxextra{hps.core.system method}}

\begin{fulllineitems}
\phantomsection\label{\detokenize{modules/system:hps.core.system.setCAMassPerResidueType}}
\pysigstartsignatures
\pysiglinewithargsret{\sphinxbfcode{\sphinxupquote{setCAMassPerResidueType}}}{}{}
\pysigstopsignatures
\sphinxAtStartPar
Sets alpha carbon atoms to their average residue mass. Used specially for
modifying alpha\sphinxhyphen{}carbon (CA) coarse\sphinxhyphen{}grained models.

\sphinxAtStartPar
Sets the masses of the alpha carbon atoms to the average mass
of its amino acid residue.
\begin{quote}\begin{description}
\sphinxlineitem{Return type}
\sphinxAtStartPar
None

\end{description}\end{quote}

\end{fulllineitems}

\index{setCARadiusPerResidueType() (hps.core.system method)@\spxentry{setCARadiusPerResidueType()}\spxextra{hps.core.system method}}

\begin{fulllineitems}
\phantomsection\label{\detokenize{modules/system:hps.core.system.setCARadiusPerResidueType}}
\pysigstartsignatures
\pysiglinewithargsret{\sphinxbfcode{\sphinxupquote{setCARadiusPerResidueType}}}{}{}
\pysigstopsignatures
\sphinxAtStartPar
Sets alpha carbon atoms to their average residue mass. Used specially for
modifying alpha\sphinxhyphen{}carbon (CA) coarse\sphinxhyphen{}grained models.

\sphinxAtStartPar
Sets the excluded volume radii of the alpha carbon atoms
to characteristic radii of their corresponding amino acid
residue.
\begin{quote}\begin{description}
\sphinxlineitem{Return type}
\sphinxAtStartPar
None

\end{description}\end{quote}

\end{fulllineitems}

\index{setCAChargePerResidueType() (hps.core.system method)@\spxentry{setCAChargePerResidueType()}\spxextra{hps.core.system method}}

\begin{fulllineitems}
\phantomsection\label{\detokenize{modules/system:hps.core.system.setCAChargePerResidueType}}
\pysigstartsignatures
\pysiglinewithargsret{\sphinxbfcode{\sphinxupquote{setCAChargePerResidueType}}}{}{}
\pysigstopsignatures
\sphinxAtStartPar
Sets the charge of the alpha carbon atoms
to characteristic charge of their corresponding amino acid
residue.
\begin{quote}\begin{description}
\sphinxlineitem{Return type}
\sphinxAtStartPar
None

\end{description}\end{quote}

\end{fulllineitems}

\index{setCAHPSPerResidueType() (hps.core.system method)@\spxentry{setCAHPSPerResidueType()}\spxextra{hps.core.system method}}

\begin{fulllineitems}
\phantomsection\label{\detokenize{modules/system:hps.core.system.setCAHPSPerResidueType}}
\pysigstartsignatures
\pysiglinewithargsret{\sphinxbfcode{\sphinxupquote{setCAHPSPerResidueType}}}{}{}
\pysigstopsignatures
\sphinxAtStartPar
Sets alpha carbon atoms to their residue hydropathy scale. Used specially for
modifying alpha\sphinxhyphen{}carbon (CA) coarse\sphinxhyphen{}grained models.

\sphinxAtStartPar
Sets the HPS model of the alpha carbon atoms using corresponding scale.
\begin{quote}\begin{description}
\sphinxlineitem{Return type}
\sphinxAtStartPar
None

\end{description}\end{quote}

\end{fulllineitems}

\index{\_setParameters() (hps.core.system static method)@\spxentry{\_setParameters()}\spxextra{hps.core.system static method}}

\begin{fulllineitems}
\phantomsection\label{\detokenize{modules/system:hps.core.system._setParameters}}
\pysigstartsignatures
\pysiglinewithargsret{\sphinxbfcode{\sphinxupquote{static\DUrole{w}{  }}}\sphinxbfcode{\sphinxupquote{\_setParameters}}}{\emph{\DUrole{n}{term}}, \emph{\DUrole{n}{parameters}}}{}
\pysigstopsignatures
\sphinxAtStartPar
General function to set up or change force field parameters.
protected method, can be called only inside class system.
\begin{quote}\begin{description}
\sphinxlineitem{Parameters}\begin{itemize}
\item {} 
\sphinxAtStartPar
\sphinxstyleliteralstrong{\sphinxupquote{term}} (\sphinxstyleliteralemphasis{\sphinxupquote{dict}}) \textendash{} Dictionary object containing the set of degrees of freedom
(DOF) to set up attributes to (e.g. \sphinxcode{\sphinxupquote{bonds}} attribute)

\item {} 
\sphinxAtStartPar
\sphinxstyleliteralstrong{\sphinxupquote{parameters}} (\sphinxstyleliteralemphasis{\sphinxupquote{integer}}\sphinxstyleliteralemphasis{\sphinxupquote{ or }}\sphinxstyleliteralemphasis{\sphinxupquote{float}}\sphinxstyleliteralemphasis{\sphinxupquote{ or }}\sphinxstyleliteralemphasis{\sphinxupquote{list}}) \textendash{} Value(s) for the specific forcefield parameters. If integer
or float, sets up the same value for all the DOF in terms.
If a list is given, sets a unique parameter for each DOF.

\end{itemize}

\sphinxlineitem{Return type}
\sphinxAtStartPar
None

\end{description}\end{quote}

\end{fulllineitems}


\end{fulllineitems}


\sphinxstepscope


\chapter{Models}
\label{\detokenize{modules/models:models}}\label{\detokenize{modules/models::doc}}
\sphinxAtStartPar
The models class contains three methods for automatic setting up predefined potentials.

\sphinxAtStartPar
It works by initializing a system class with the necessary force field parameters.


\section{Coarse grained, alpha\sphinxhyphen{}carbon (CA), model}
\label{\detokenize{modules/models:coarse-grained-alpha-carbon-ca-model}}
\sphinxAtStartPar
The coarse grained method represents the protein system as beads centered at the alpha carbons of each residue in the protein.

\sphinxAtStartPar
It uses harmonic potentials to hold the covalent connectivity and geometry of the beads.

\sphinxAtStartPar
Torsional geometries are modeled with a periodic torsion potential.

\sphinxAtStartPar
Native contacts are represented through the use of Lennard\sphinxhyphen{}Jones potentials that allow to form and break non\sphinxhyphen{}bonded interactions, permitting complete and local unfolding of the structures.

\sphinxAtStartPar
To create a CA model, call:
\sphinxcode{\sphinxupquote{hps.models.getCAModel(pdb\_file, hps\_scale)}}

\sphinxAtStartPar
Here, pdb\_file is the path to the PDB format structure of the protein.
hps\_scale is hydropathy scale that are going to be used. \sphinxcode{\sphinxupquote{urry}} or \sphinxcode{\sphinxupquote{kr}}

\sphinxAtStartPar
The force field equations are:
\begin{equation*}
\begin{split}H_A = \sum_{bonds}V_{bond}+\sum_{i,j}\Phi_{ij}^{vdw}+\sum_{i,j}\Phi_{i,j}^{el}\end{split}
\end{equation*}
\sphinxAtStartPar
If hps\_ss model is used, the Hamiltonian is:
\begin{equation*}
\begin{split}H_{hps-ss} = \sum_{bonds}V_{bond}+\sum_{angle}V_{angle}+\sum_{torsion}V_{torsion}+\sum_{i,j}\Phi_{ij}^{vdw}+\sum_{i,j}\Phi_{i,j}^{el}\end{split}
\end{equation*}

\section{The Bonded potential:}
\label{\detokenize{modules/models:the-bonded-potential}}\begin{equation*}
\begin{split}V_{bond} = \frac{k_b}{2}(r-r_0)^2\end{split}
\end{equation*}
\sphinxAtStartPar
Here the default values are:
\begin{quote}

\sphinxAtStartPar
\(k_b= 8368 kJ/(mol \times nm^2),\\
r_0=0.382 nm\)
\end{quote}


\section{Angle Potential}
\label{\detokenize{modules/models:angle-potential}}\begin{equation*}
\begin{split}U_{angle}(\theta) = \frac{-1}{\gamma}
\ln \left[ e^{ -\gamma[ k_{\alpha} (\theta-\theta_{\alpha})^2+\epsilon_{\alpha} ]} +e^{ -\gamma k_{\beta} (\theta-\theta_{\beta})^2 } \right]\end{split}
\end{equation*}
\sphinxAtStartPar
Parameters:
\begin{quote}

\sphinxAtStartPar
\(\gamma = 0.1 mol/kcal,\\
\epsilon_{\alpha}=4.3 kcal/mol,\\
\theta_{\alpha}=1.6 rad, \\
\theta_{\beta}=2.27 rad\)
\end{quote}


\section{Torsion Potential}
\label{\detokenize{modules/models:torsion-potential}}\begin{align*}\!\begin{aligned}
U_{torsion}(\theta) = -\ln\left[ U_{torsion, \alpha}(\theta, \epsilon_d) + U_{torsion, \beta}(\theta, \epsilon_d)\right]\\
U_{torsion, \alpha}(\theta, \epsilon_d)  = e^{-k_{\alpha, 1}(\theta-\theta_{\alpha,1})^2-\epsilon_d}
                                            + e^{-k_{\alpha, 2}(\theta-\theta_{\alpha,2})^4 + e_0}
                                            + e^{-k_{\alpha, 2}(\theta-\theta_{\alpha,2}+2\pi)^4 + e_0}\\
U_{torsion, \beta}(\theta, \epsilon_d) = e^{-k_{\beta,1}(\theta-\theta_{\beta,1})^2+e_1+\epsilon_d}
                                       + e^{-k_{\beta,1}(\theta-\theta_{\beta,1}-2\pi)^2+e_1+\epsilon_d} \\
                                       + e^{-k_{\beta,2}(\theta-\theta_{\beta,2})^4+e_2}
                                       + e^{-k_{\beta,2}(\theta-\theta_{\beta,2}-2\pi)^4+e_2}\\
\end{aligned}\end{align*}
\sphinxAtStartPar
Parameters:
\begin{quote}

\sphinxAtStartPar
\(k_{\alpha,1}=11.4 kcal/(mol \times rad^2),\\
k_{\alpha,2}=0.15 kcal/(mol\times rad^4),\\
\theta_{\alpha,1} = 0.9 rad,\\
\theta_{\alpha,2}=1.02 rad,\\
e_0 = 0.27 kcal/mol,\\
k_{\beta,1}=1.8kcal/(mol \times rad^2),\\
k_{\beta,2}=0.65kcal/(mol\times rad^4),\\
\theta_{\beta,1}=-1.55 rad,\\
\theta_{\beta,2}=-2.5 rad,\\
e_1 = 0.14 kcal/mol,\\
e_2 = 0.4 kcal/mol\)
\end{quote}


\section{The Pairwise potential:}
\label{\detokenize{modules/models:the-pairwise-potential}}\begin{align*}\!\begin{aligned}
\Phi_{i,j}^{vdw}(r) = step(2^{1/6}\sigma_{ij}-r) \times \left( 4\epsilon\left[\left(\frac{\sigma_{ij}}{r}\right)^{12}- \left(\frac{\sigma_{ij}}{r}\right)^{6}\right]+(1-\mu\times\lambda_{ij}^{0}+\Delta)\times\epsilon\right)\\
+ \left[1-step(2^{1/6}\sigma_{ij}-r)\right]\times\left[(\mu \lambda_{ij}^{0}-\Delta)\times 4\epsilon \left[\left(\frac{\sigma_{ij}}{r}\right)^{12}-\left(\frac{\sigma_{ij}}{r}\right)^6\right]\right]\\
\end{aligned}\end{align*}
\sphinxAtStartPar
Since the step function behaves like: \sphinxcode{\sphinxupquote{step(x) = 0 if x \textless{} 0,and =1 otherwise}}, we can separate in multiple cases for short likes following:
\begin{align*}\!\begin{aligned}
\Phi_{i,j}^{vdw}(r) =  4\epsilon \left[\left(\frac{\sigma_{ij}}{r}\right)^{12}-\left(\frac{\sigma_{ij}}{r}\right)^{6}\right]+(1-\mu     \times\lambda_{ij}^{0}+\Delta)  \times\epsilon, r\le 2^{1/6}\sigma_{ij}\\
\Phi_{i,j}^{vdw}(r) = (\mu\times\lambda_{ij}^{0}-\Delta) \times \left( 4\epsilon \left[\left(\frac{\sigma_{ij}}{r}\right)^{12}-\left(\frac{\sigma_{ij}}{r}\right)^{6}\right]\right), r > 2^{1/6}\sigma_{ij}\\
\end{aligned}\end{align*}
\sphinxAtStartPar
where, \(\sigma_{i,j}=\frac{\sigma_i+\sigma_j}{2}\): is the vdW radius interaction of interacting beads

\sphinxAtStartPar
\(\lambda_{ij}^{0}=\frac{\lambda_i+\lambda_j}{2}\): hydropathy scale interaction of residues

\sphinxAtStartPar
\(\mu, \Delta\): are the only free parameters in the model. In Jeetain Mittal(2021) Protein Science, he simulated for 42 IDP proteins and fit Rg vs experimental values.

\sphinxAtStartPar
In the current implementation, hydropathy scales are taken from Urry model, \((\mu, \Delta) = (1, 0.08)\)

\sphinxAtStartPar
Nonbonded exclusion rule is \sphinxcode{\sphinxupquote{1\sphinxhyphen{}2}}, for hps\_kr and hps\_urry which we only exclude pair of atoms in bonded.
while it is \sphinxcode{\sphinxupquote{1\sphinxhyphen{}4}} for hps\sphinxhyphen{}ss, which we exclude 3 bonds.

\sphinxAtStartPar
The cut\sphinxhyphen{}off distance for Lennard\sphinxhyphen{}Jone potential: \(2.0 nm\)


\section{The Debye\sphinxhyphen{}Huckle potential has following form:}
\label{\detokenize{modules/models:the-debye-huckle-potential-has-following-form}}\begin{equation*}
\begin{split}\Phi_{ij}^{el}(r) = \frac{q_{i}q_{j}}{4\pi\epsilon_0 D r}e^{-\kappa r}\end{split}
\end{equation*}
\sphinxAtStartPar
where, \(q_i, q_j\) are charge of residues \(i, j\)

\sphinxAtStartPar
\(\epsilon_0\): Vacuum permitivity. For convenient, we precalculated the electric conversion factor
\(\frac{1}{4\pi\epsilon_0}= 138.935 485(9) kJ \times mol^{−1} \times nm \times e^{−2}\).

\sphinxAtStartPar
\(D\): dielectric constant, at 100mM mono\sphinxhyphen{}valence salt (NaCl), it takes values of 80.
The dielectric constant here is fixed, but it can be temperature dependent as the function:
\(\frac{5321}{T}+233.76-0.9297T+0.1417\times 10^{-2}\times T^2 - 0.8292\times 10^{-6}\times T^3\)

\sphinxAtStartPar
\(\kappa\): inverse Debye length, at 100mM NaCl has values of \(1 nm^{-1}\)

\sphinxAtStartPar
The cut\sphinxhyphen{}off distance for Electrostatics interactions: \(3.5 nm\)
\index{models (class in hps.core)@\spxentry{models}\spxextra{class in hps.core}}

\begin{fulllineitems}
\phantomsection\label{\detokenize{modules/models:hps.core.models}}
\pysigstartsignatures
\pysigline{\sphinxbfcode{\sphinxupquote{class\DUrole{w}{  }}}\sphinxcode{\sphinxupquote{hps.core.}}\sphinxbfcode{\sphinxupquote{models}}}
\pysigstopsignatures
\sphinxAtStartPar
A class to hold functions for the automated generation of default hps models.
\index{\_\_init\_\_() (hps.core.models method)@\spxentry{\_\_init\_\_()}\spxextra{hps.core.models method}}

\begin{fulllineitems}
\phantomsection\label{\detokenize{modules/models:hps.core.models.__init__}}
\pysigstartsignatures
\pysiglinewithargsret{\sphinxbfcode{\sphinxupquote{\_\_init\_\_}}}{}{}
\pysigstopsignatures
\end{fulllineitems}

\index{buildHPSModel() (hps.core.models static method)@\spxentry{buildHPSModel()}\spxextra{hps.core.models static method}}

\begin{fulllineitems}
\phantomsection\label{\detokenize{modules/models:hps.core.models.buildHPSModel}}
\pysigstartsignatures
\pysiglinewithargsret{\sphinxbfcode{\sphinxupquote{static\DUrole{w}{  }}}\sphinxbfcode{\sphinxupquote{buildHPSModel}}}{\emph{\DUrole{n}{structure\_file}\DUrole{p}{:}\DUrole{w}{  }\DUrole{n}{str}}, \emph{\DUrole{n}{minimize}\DUrole{p}{:}\DUrole{w}{  }\DUrole{n}{bool}\DUrole{w}{  }\DUrole{o}{=}\DUrole{w}{  }\DUrole{default_value}{False}}, \emph{\DUrole{n}{hps\_scale}\DUrole{p}{:}\DUrole{w}{  }\DUrole{n}{str}\DUrole{w}{  }\DUrole{o}{=}\DUrole{w}{  }\DUrole{default_value}{\textquotesingle{}hps\_urry\textquotesingle{}}}, \emph{\DUrole{n}{box\_dimension}\DUrole{p}{:}\DUrole{w}{  }\DUrole{n}{Optional\DUrole{p}{{[}}Any\DUrole{p}{{]}}}\DUrole{w}{  }\DUrole{o}{=}\DUrole{w}{  }\DUrole{default_value}{None}}}{}
\pysigstopsignatures
\sphinxAtStartPar
Creates an alpha\sphinxhyphen{}carbon only \sphinxcode{\sphinxupquote{hpsOpenMM}} system class object with default
initialized parameters.

\sphinxAtStartPar
Initializes a coarse\sphinxhyphen{}grained, carbon alpha (CA), hpsOpenMM system class
from a structure and a contact file defining the native contacts for the
coarse grained model.

\sphinxAtStartPar
The system creation steps are:
\begin{enumerate}
\sphinxsetlistlabels{\arabic}{enumi}{enumii}{}{)}%
\item {} 
\sphinxAtStartPar
Add the geometrical parameters for the model.

\item {} 
\sphinxAtStartPar
Add the default force field parameters for the model.

\item {} 
\sphinxAtStartPar
Create the default force objects.

\item {} 
\sphinxAtStartPar
Create the OpenMM system class.

\end{enumerate}

\sphinxAtStartPar
The method can be used to generate an initialized hpsOpenMM system class, that only
contains the geometrical parameters, by passing the option default\_parameters as False.
\begin{quote}\begin{description}
\sphinxlineitem{Parameters}\begin{itemize}
\item {} 
\sphinxAtStartPar
\sphinxstyleliteralstrong{\sphinxupquote{structure\_file}} (\sphinxstyleliteralemphasis{\sphinxupquote{string}}\sphinxstyleliteralemphasis{\sphinxupquote{ {[}}}\sphinxstyleliteralemphasis{\sphinxupquote{requires}}\sphinxstyleliteralemphasis{\sphinxupquote{{]}}}) \textendash{} Path to the input structure file.

\item {} 
\sphinxAtStartPar
\sphinxstyleliteralstrong{\sphinxupquote{minimize}} (\sphinxstyleliteralemphasis{\sphinxupquote{boolean}}\sphinxstyleliteralemphasis{\sphinxupquote{ (}}\sphinxstyleliteralemphasis{\sphinxupquote{False}}\sphinxstyleliteralemphasis{\sphinxupquote{)}}) \textendash{} If True the initial structure will undergo the energy minimization.

\item {} 
\sphinxAtStartPar
\sphinxstyleliteralstrong{\sphinxupquote{hps\_scale}} (\sphinxstyleliteralemphasis{\sphinxupquote{string}}\sphinxstyleliteralemphasis{\sphinxupquote{ {[}}}\sphinxstyleliteralemphasis{\sphinxupquote{Optional}}\sphinxstyleliteralemphasis{\sphinxupquote{, }}\sphinxstyleliteralemphasis{\sphinxupquote{hps\_urry}}\sphinxstyleliteralemphasis{\sphinxupquote{{]}}}) \textendash{} \begin{description}
\sphinxlineitem{HPS scale. There are three options correspond to two scale:}\begin{itemize}
\item {} 
\sphinxAtStartPar
’hps\_urry’: using Urry scale (default).

\item {} 
\sphinxAtStartPar
’hps\_ss’: hps\_urry with angle and torsion potential.

\item {} 
\sphinxAtStartPar
’hps\_kr’: using Kapcha\sphinxhyphen{}Rossy scale.

\end{itemize}

\end{description}


\item {} 
\sphinxAtStartPar
\sphinxstyleliteralstrong{\sphinxupquote{box\_dimension}} (\sphinxstyleliteralemphasis{\sphinxupquote{float}}\sphinxstyleliteralemphasis{\sphinxupquote{ or }}\sphinxstyleliteralemphasis{\sphinxupquote{array}}\sphinxstyleliteralemphasis{\sphinxupquote{ (}}\sphinxstyleliteralemphasis{\sphinxupquote{None}}\sphinxstyleliteralemphasis{\sphinxupquote{)}}) \textendash{} If box\_dimension is supplied, then will use PBC.
if float is given, then use cubic box
if an array of (3,1) is given, then use rectangular box with the given dimension
if not specify: do not use PBC

\end{itemize}

\sphinxlineitem{Returns}
\sphinxAtStartPar
\sphinxstylestrong{hps} \textendash{} Initialized hpsOpenMM.system class with default options for defining
a coarse\sphinxhyphen{}grained CA force field.

\sphinxlineitem{Return type}
\sphinxAtStartPar
\sphinxcode{\sphinxupquote{hpsOpenMM.system}}

\end{description}\end{quote}

\end{fulllineitems}


\end{fulllineitems}


\sphinxstepscope


\chapter{Dynamics}
\label{\detokenize{modules/dynamics:dynamics}}\label{\detokenize{modules/dynamics::doc}}\begin{description}
\sphinxlineitem{Dynamics class contains two main functions: read config file and run simulation.}
\sphinxAtStartPar
User only need to provide config file, e.g md.ini and specify parameters control simulation there.

\end{description}
\index{Dynamics (class in hps.dynamics)@\spxentry{Dynamics}\spxextra{class in hps.dynamics}}

\begin{fulllineitems}
\phantomsection\label{\detokenize{modules/dynamics:hps.dynamics.Dynamics}}
\pysigstartsignatures
\pysiglinewithargsret{\sphinxbfcode{\sphinxupquote{class\DUrole{w}{  }}}\sphinxcode{\sphinxupquote{hps.dynamics.}}\sphinxbfcode{\sphinxupquote{Dynamics}}}{\emph{\DUrole{n}{config\_file}}}{}
\pysigstopsignatures
\sphinxAtStartPar
Dynamics class contains two main functions: read config file and run simulation.
User only need to provide config file, e.g md.ini and specify parameters control simulation there.
\begin{quote}\begin{description}
\sphinxlineitem{Parameters}
\sphinxAtStartPar
\sphinxstyleliteralstrong{\sphinxupquote{config\_file}} (\sphinxstyleliteralemphasis{\sphinxupquote{str}}) \textendash{} control parameters for simulation

\end{description}\end{quote}
\index{md\_steps (hps.dynamics.Dynamics attribute)@\spxentry{md\_steps}\spxextra{hps.dynamics.Dynamics attribute}}

\begin{fulllineitems}
\phantomsection\label{\detokenize{modules/dynamics:hps.dynamics.Dynamics.md_steps}}
\pysigstartsignatures
\pysigline{\sphinxbfcode{\sphinxupquote{md\_steps}}}
\pysigstopsignatures
\sphinxAtStartPar
Number of steps to perform molecular dynamics simulation
\begin{quote}\begin{description}
\sphinxlineitem{Type}
\sphinxAtStartPar
int {[}1, steps{]}

\end{description}\end{quote}

\end{fulllineitems}

\index{dt (hps.dynamics.Dynamics attribute)@\spxentry{dt}\spxextra{hps.dynamics.Dynamics attribute}}

\begin{fulllineitems}
\phantomsection\label{\detokenize{modules/dynamics:hps.dynamics.Dynamics.dt}}
\pysigstartsignatures
\pysigline{\sphinxbfcode{\sphinxupquote{dt}}}
\pysigstopsignatures
\sphinxAtStartPar
time step for integration
\begin{quote}\begin{description}
\sphinxlineitem{Type}
\sphinxAtStartPar
float {[}0.01, ps{]}

\end{description}\end{quote}

\end{fulllineitems}

\index{nstxout (hps.dynamics.Dynamics attribute)@\spxentry{nstxout}\spxextra{hps.dynamics.Dynamics attribute}}

\begin{fulllineitems}
\phantomsection\label{\detokenize{modules/dynamics:hps.dynamics.Dynamics.nstxout}}
\pysigstartsignatures
\pysigline{\sphinxbfcode{\sphinxupquote{nstxout}}}
\pysigstopsignatures
\sphinxAtStartPar
number of steps that elapse between writing coordinates to output trajectory file,
the last coordinates are always written
\begin{quote}\begin{description}
\sphinxlineitem{Type}
\sphinxAtStartPar
int {[}1, steps{]}

\end{description}\end{quote}

\end{fulllineitems}

\index{nstlog (hps.dynamics.Dynamics attribute)@\spxentry{nstlog}\spxextra{hps.dynamics.Dynamics attribute}}

\begin{fulllineitems}
\phantomsection\label{\detokenize{modules/dynamics:hps.dynamics.Dynamics.nstlog}}
\pysigstartsignatures
\pysigline{\sphinxbfcode{\sphinxupquote{nstlog}}}
\pysigstopsignatures
\sphinxAtStartPar
number of steps that elapse between writing energies to the log file, the last energies are always written
\begin{quote}\begin{description}
\sphinxlineitem{Type}
\sphinxAtStartPar
int {[}1, steps{]}

\end{description}\end{quote}

\end{fulllineitems}

\index{nstcomm (hps.dynamics.Dynamics attribute)@\spxentry{nstcomm}\spxextra{hps.dynamics.Dynamics attribute}}

\begin{fulllineitems}
\phantomsection\label{\detokenize{modules/dynamics:hps.dynamics.Dynamics.nstcomm}}
\pysigstartsignatures
\pysigline{\sphinxbfcode{\sphinxupquote{nstcomm}}}
\pysigstopsignatures
\sphinxAtStartPar
frequency for center of mass motion removal
\begin{quote}\begin{description}
\sphinxlineitem{Type}
\sphinxAtStartPar
int {[}100, steps{]}

\end{description}\end{quote}

\end{fulllineitems}

\index{model (hps.dynamics.Dynamics attribute)@\spxentry{model}\spxextra{hps.dynamics.Dynamics attribute}}

\begin{fulllineitems}
\phantomsection\label{\detokenize{modules/dynamics:hps.dynamics.Dynamics.model}}
\pysigstartsignatures
\pysigline{\sphinxbfcode{\sphinxupquote{model}}}
\pysigstopsignatures
\sphinxAtStartPar
Hydropathy scale
\begin{quote}\begin{description}
\sphinxlineitem{Type}
\sphinxAtStartPar
str {[}‘hps\_urry’{]}

\end{description}\end{quote}

\end{fulllineitems}

\index{tcoupl (hps.dynamics.Dynamics attribute)@\spxentry{tcoupl}\spxextra{hps.dynamics.Dynamics attribute}}

\begin{fulllineitems}
\phantomsection\label{\detokenize{modules/dynamics:hps.dynamics.Dynamics.tcoupl}}
\pysigstartsignatures
\pysigline{\sphinxbfcode{\sphinxupquote{tcoupl}}}
\pysigstopsignatures
\sphinxAtStartPar
Using temperature coupling.
\begin{quote}\begin{description}
\sphinxlineitem{Type}
\sphinxAtStartPar
bool

\end{description}\end{quote}

\end{fulllineitems}

\index{ref\_t (hps.dynamics.Dynamics attribute)@\spxentry{ref\_t}\spxextra{hps.dynamics.Dynamics attribute}}

\begin{fulllineitems}
\phantomsection\label{\detokenize{modules/dynamics:hps.dynamics.Dynamics.ref_t}}
\pysigstartsignatures
\pysigline{\sphinxbfcode{\sphinxupquote{ref\_t}}}
\pysigstopsignatures
\sphinxAtStartPar
reference temperature for coupling
\begin{quote}\begin{description}
\sphinxlineitem{Type}
\sphinxAtStartPar
float {[}Kelvin{]}

\end{description}\end{quote}

\end{fulllineitems}

\index{tau\_t (hps.dynamics.Dynamics attribute)@\spxentry{tau\_t}\spxextra{hps.dynamics.Dynamics attribute}}

\begin{fulllineitems}
\phantomsection\label{\detokenize{modules/dynamics:hps.dynamics.Dynamics.tau_t}}
\pysigstartsignatures
\pysigline{\sphinxbfcode{\sphinxupquote{tau\_t}}}
\pysigstopsignatures
\sphinxAtStartPar
ime constant for temperature coupling
\begin{quote}\begin{description}
\sphinxlineitem{Type}
\sphinxAtStartPar
float {[}ps{]}

\end{description}\end{quote}

\end{fulllineitems}

\index{pcoupl (hps.dynamics.Dynamics attribute)@\spxentry{pcoupl}\spxextra{hps.dynamics.Dynamics attribute}}

\begin{fulllineitems}
\phantomsection\label{\detokenize{modules/dynamics:hps.dynamics.Dynamics.pcoupl}}
\pysigstartsignatures
\pysigline{\sphinxbfcode{\sphinxupquote{pcoupl}}}
\pysigstopsignatures
\sphinxAtStartPar
Pressure coupling
\begin{quote}\begin{description}
\sphinxlineitem{Type}
\sphinxAtStartPar
bool

\end{description}\end{quote}

\end{fulllineitems}

\index{ref\_p (hps.dynamics.Dynamics attribute)@\spxentry{ref\_p}\spxextra{hps.dynamics.Dynamics attribute}}

\begin{fulllineitems}
\phantomsection\label{\detokenize{modules/dynamics:hps.dynamics.Dynamics.ref_p}}
\pysigstartsignatures
\pysigline{\sphinxbfcode{\sphinxupquote{ref\_p}}}
\pysigstopsignatures
\sphinxAtStartPar
The reference pressure for coupling.
\begin{quote}\begin{description}
\sphinxlineitem{Type}
\sphinxAtStartPar
float {[}bar{]}

\end{description}\end{quote}

\end{fulllineitems}

\index{frequency\_p (hps.dynamics.Dynamics attribute)@\spxentry{frequency\_p}\spxextra{hps.dynamics.Dynamics attribute}}

\begin{fulllineitems}
\phantomsection\label{\detokenize{modules/dynamics:hps.dynamics.Dynamics.frequency_p}}
\pysigstartsignatures
\pysigline{\sphinxbfcode{\sphinxupquote{frequency\_p}}}
\pysigstopsignatures
\sphinxAtStartPar
The frequency for coupling the pressure.
\begin{quote}\begin{description}
\sphinxlineitem{Type}
\sphinxAtStartPar
int {[}25, steps{]}

\end{description}\end{quote}

\end{fulllineitems}

\index{pbc (hps.dynamics.Dynamics attribute)@\spxentry{pbc}\spxextra{hps.dynamics.Dynamics attribute}}

\begin{fulllineitems}
\phantomsection\label{\detokenize{modules/dynamics:hps.dynamics.Dynamics.pbc}}
\pysigstartsignatures
\pysigline{\sphinxbfcode{\sphinxupquote{pbc}}}
\pysigstopsignatures
\sphinxAtStartPar
Use periodic boundary conditions.
\begin{quote}\begin{description}
\sphinxlineitem{Type}
\sphinxAtStartPar
bool

\end{description}\end{quote}

\end{fulllineitems}

\index{box\_dimension (hps.dynamics.Dynamics attribute)@\spxentry{box\_dimension}\spxextra{hps.dynamics.Dynamics attribute}}

\begin{fulllineitems}
\phantomsection\label{\detokenize{modules/dynamics:hps.dynamics.Dynamics.box_dimension}}
\pysigstartsignatures
\pysigline{\sphinxbfcode{\sphinxupquote{box\_dimension}}}
\pysigstopsignatures
\sphinxAtStartPar
Box dimension defined the unit cell, better to use rectangular for simplicity
\begin{quote}\begin{description}
\sphinxlineitem{Type}
\sphinxAtStartPar
float or list of float

\end{description}\end{quote}

\end{fulllineitems}

\index{protein\_code (hps.dynamics.Dynamics attribute)@\spxentry{protein\_code}\spxextra{hps.dynamics.Dynamics attribute}}

\begin{fulllineitems}
\phantomsection\label{\detokenize{modules/dynamics:hps.dynamics.Dynamics.protein_code}}
\pysigstartsignatures
\pysigline{\sphinxbfcode{\sphinxupquote{protein\_code}}}
\pysigstopsignatures
\sphinxAtStartPar
Prefix to write output file based on this parameter
\begin{quote}\begin{description}
\sphinxlineitem{Type}
\sphinxAtStartPar
str

\end{description}\end{quote}

\end{fulllineitems}

\index{checkpoint (hps.dynamics.Dynamics attribute)@\spxentry{checkpoint}\spxextra{hps.dynamics.Dynamics attribute}}

\begin{fulllineitems}
\phantomsection\label{\detokenize{modules/dynamics:hps.dynamics.Dynamics.checkpoint}}
\pysigstartsignatures
\pysigline{\sphinxbfcode{\sphinxupquote{checkpoint}}}
\pysigstopsignatures
\sphinxAtStartPar
Checkpoint file name
\begin{quote}\begin{description}
\sphinxlineitem{Type}
\sphinxAtStartPar
str

\end{description}\end{quote}

\end{fulllineitems}

\index{pdb\_file (hps.dynamics.Dynamics attribute)@\spxentry{pdb\_file}\spxextra{hps.dynamics.Dynamics attribute}}

\begin{fulllineitems}
\phantomsection\label{\detokenize{modules/dynamics:hps.dynamics.Dynamics.pdb_file}}
\pysigstartsignatures
\pysigline{\sphinxbfcode{\sphinxupquote{pdb\_file}}}
\pysigstopsignatures
\sphinxAtStartPar
Input structure read to generate model.
\begin{quote}\begin{description}
\sphinxlineitem{Type}
\sphinxAtStartPar
str

\end{description}\end{quote}

\end{fulllineitems}

\index{device (hps.dynamics.Dynamics attribute)@\spxentry{device}\spxextra{hps.dynamics.Dynamics attribute}}

\begin{fulllineitems}
\phantomsection\label{\detokenize{modules/dynamics:hps.dynamics.Dynamics.device}}
\pysigstartsignatures
\pysigline{\sphinxbfcode{\sphinxupquote{device}}}
\pysigstopsignatures
\sphinxAtStartPar
Device to perform simulation {[}GPU/CPU{]} if CPU is used, then need to provide number of threads to run simulation.
\begin{quote}\begin{description}
\sphinxlineitem{Type}
\sphinxAtStartPar
str

\end{description}\end{quote}

\end{fulllineitems}

\index{ppn (hps.dynamics.Dynamics attribute)@\spxentry{ppn}\spxextra{hps.dynamics.Dynamics attribute}}

\begin{fulllineitems}
\phantomsection\label{\detokenize{modules/dynamics:hps.dynamics.Dynamics.ppn}}
\pysigstartsignatures
\pysigline{\sphinxbfcode{\sphinxupquote{ppn}}}
\pysigstopsignatures
\sphinxAtStartPar
In case simulation is run on CPU, use this parameter to control the number of threads to run simulation.
\begin{quote}\begin{description}
\sphinxlineitem{Type}
\sphinxAtStartPar
int {[}1, cores{]}

\end{description}\end{quote}

\end{fulllineitems}

\index{restart (hps.dynamics.Dynamics attribute)@\spxentry{restart}\spxextra{hps.dynamics.Dynamics attribute}}

\begin{fulllineitems}
\phantomsection\label{\detokenize{modules/dynamics:hps.dynamics.Dynamics.restart}}
\pysigstartsignatures
\pysigline{\sphinxbfcode{\sphinxupquote{restart}}}
\pysigstopsignatures
\sphinxAtStartPar
If simulation run from beginning or restart from checkpoint.
\begin{quote}\begin{description}
\sphinxlineitem{Type}
\sphinxAtStartPar
bool {[}No{]}

\end{description}\end{quote}

\end{fulllineitems}

\index{minimize (hps.dynamics.Dynamics attribute)@\spxentry{minimize}\spxextra{hps.dynamics.Dynamics attribute}}

\begin{fulllineitems}
\phantomsection\label{\detokenize{modules/dynamics:hps.dynamics.Dynamics.minimize}}
\pysigstartsignatures
\pysigline{\sphinxbfcode{\sphinxupquote{minimize}}}
\pysigstopsignatures
\sphinxAtStartPar
If simulation run from beginning then need to perform energy minimization. If simulation restarted, this
parameters will be override to False.
\begin{quote}\begin{description}
\sphinxlineitem{Type}
\sphinxAtStartPar
bool

\end{description}\end{quote}

\end{fulllineitems}

\index{\_\_init\_\_() (hps.dynamics.Dynamics method)@\spxentry{\_\_init\_\_()}\spxextra{hps.dynamics.Dynamics method}}

\begin{fulllineitems}
\phantomsection\label{\detokenize{modules/dynamics:hps.dynamics.Dynamics.__init__}}
\pysigstartsignatures
\pysiglinewithargsret{\sphinxbfcode{\sphinxupquote{\_\_init\_\_}}}{\emph{\DUrole{n}{config\_file}}}{}
\pysigstopsignatures
\end{fulllineitems}

\index{read\_config() (hps.dynamics.Dynamics method)@\spxentry{read\_config()}\spxextra{hps.dynamics.Dynamics method}}

\begin{fulllineitems}
\phantomsection\label{\detokenize{modules/dynamics:hps.dynamics.Dynamics.read_config}}
\pysigstartsignatures
\pysiglinewithargsret{\sphinxbfcode{\sphinxupquote{read\_config}}}{\emph{\DUrole{n}{config\_file}}}{}
\pysigstopsignatures
\sphinxAtStartPar
Read simulation control parameters from config file {\color{red}\bfseries{}*}.ini into class attributes.
\begin{description}
\sphinxlineitem{TODO: check parameters in control file more carefully.}
\sphinxAtStartPar
Raise error and exit immediately if something wrong.

\end{description}

\end{fulllineitems}


\end{fulllineitems}



\chapter{Changelog}
\label{\detokenize{index:changelog}}

\chapter{About}
\label{\detokenize{index:about}}

\renewcommand{\indexname}{Python Module Index}
\begin{sphinxtheindex}
\let\bigletter\sphinxstyleindexlettergroup
\bigletter{h}
\item\relax\sphinxstyleindexentry{hps.parameters.model\_parameters}\sphinxstyleindexpageref{modules/parameters:\detokenize{module-hps.parameters.model_parameters}}
\end{sphinxtheindex}

\renewcommand{\indexname}{Index}
\printindex
\end{document}